\documentclass{amsart}

\usepackage[T1]{fontenc}
\usepackage{amsmath, amsfonts, amssymb, amsthm, mathrsfs, wasysym, graphics, graphicx, xcolor, url, hyperref, hypcap, xargs, multicol, pdflscape, multirow, hvfloat, array, ae, aecompl, pifont, mathtools, a4wide, float, blkarray, overpic, nicefrac}
\usepackage[shortlabels, inline]{enumitem}%shortlabels to have same syntax as enumerate package, inline gives inline option with *
\usepackage{bbm}%allows for \mathbbm{1}
\usepackage[noabbrev,capitalise]{cleveref}
\usepackage[normalem]{ulem}
\usepackage{marginnote}
\hypersetup{colorlinks=true, citecolor=darkblue, linkcolor=darkblue}
\usepackage[all]{xy}
\usepackage{tikz}
\usepackage{tikz-cd}
%\usepackage{tkz-graph}
\usetikzlibrary{trees, decorations, decorations.pathmorphing, decorations.markings, decorations.shapes, shapes, arrows, matrix, calc, fit, intersections, patterns, angles}
\graphicspath{{figures/}{figures/diagonals/}{figures/walks/}{figures/tubes/}{figures/blocks/}}
\makeatletter\def\input@path{{figures/}}\makeatother
\usepackage{caption}
\captionsetup{width=\textwidth}
\usepackage[export]{adjustbox}

%%%%%%%%%%%%%%%%%%%%%%%%%%%%%%%%%%%%%%

% STANDARD

% theorems
\newtheorem{theorem}{Theorem}[section]
\newtheorem{corollary}[theorem]{Corollary}
\newtheorem{proposition}[theorem]{Proposition}
\newtheorem{lemma}[theorem]{Lemma}
\newtheorem{conjecture}[theorem]{Conjecture}
\newtheorem*{theorem*}{Theorem}%[section]

\theoremstyle{definition}
\newtheorem{definition}[theorem]{Definition}
\newtheorem{example}[theorem]{Example}
\newtheorem{remark}[theorem]{Remark}
\newtheorem{question}[theorem]{Question}
\newtheorem{notation}[theorem]{Notation}
\newtheorem{assumption}[theorem]{Assumption}
\newtheorem{convention}[theorem]{Convention}

\crefname{equation}{Equation}{Equations}

% math special letters
\newcommand{\R}{\mathbb{R}} % reals
\newcommand{\Q}{\mathbb{Q}} % rationals
\newcommand{\N}{\mathbb{N}} % naturals
\newcommand{\Z}{\mathbb{Z}} % integers
\newcommand{\C}{\mathbb{C}} % complex
\newcommand{\I}{\mathbb{I}} % set of integers
\newcommand{\K}{k} % field
\newcommand{\bb}[1]{{\mathbb{#1}}} % mathbb letters
\newcommand{\f}[1]{{\mathfrak{#1}}} % mathfrak letters
\renewcommand{\c}[1]{{\mathcal{#1}}} % call letters
\renewcommand{\b}[1]{{\boldsymbol{#1}}} % bold letters
\newcommand{\h}{\widehat} % hat letters

% math commands
\newcommand{\set}[2]{\left\{ #1 \;\middle|\; #2 \right\}} % set notation
\newcommand{\bigset}[2]{\big\{ #1 \;\big|\; #2 \big\}} % big set notation
\newcommand{\Bigset}[2]{\Big\{ #1 \;\Big|\; #2 \Big\}} % Big set notation
\newcommand{\setangle}[2]{\left\langle #1 \;\middle|\; #2 \right\rangle} % set notation
\newcommand{\ssm}{\smallsetminus} % small set minus
\newcommand{\dotprod}[2]{\langle \, #1 \; | \; #2 \, \rangle} % dot product
\newcommand{\bigdotprod}[2]{\big\langle \, #1 \; \big| \; #2 \, \big\rangle} % dot product
\newcommand{\symdif}{\,\triangle\,} % symmetric difference
\newcommand{\one}{\mathbbm{1}} % the all one vector
\newcommandx{\ones}[1][1=n]{\one_{#1}} % the all one vector of length n
\newcommand{\eqdef}{\mbox{\,\raisebox{0.2ex}{\scriptsize\ensuremath{\mathrm:}}\ensuremath{=}\,}} % :=
\newcommand{\defeq}{\mbox{~\ensuremath{=}\raisebox{0.2ex}{\scriptsize\ensuremath{\mathrm:}} }} % =:
\newcommand{\simplex}{\triangle} % simplex
\renewcommand{\implies}{\Rightarrow} % imply sign
\newcommand{\transpose}[1]{{#1}^T} % transpose matrix
\newcommand{\truth}[1]{\left[ #1 \right]} % truth (kronecker delta)

% operators
\DeclareMathOperator{\conv}{conv} % convex hull
\DeclareMathOperator{\vect}{vect} % linear span
\DeclareMathOperator{\cone}{cone} % cone hull

% others
\newcommand{\ie}{\textit{i.e.}~} % id est
\newcommand{\eg}{\textit{e.g.}~} % exempli gratia
\newcommand{\Eg}{\textit{E.g.}~} % exempli gratia
\newcommand{\apriori}{\textit{a priori}} % a priori
\newcommand{\viceversa}{\textit{vice versa}} % vice versa
\newcommand{\versus}{\textit{vs.}~} % versus
\newcommand{\aka}{\textit{a.k.a.}~} % also known as
\newcommand{\perse}{\textit{per se}} % per se
\newcommand{\ordinal}{\textsuperscript{th}} % th for ordinals
\newcommand{\ordinalst}{\textsuperscript{st}} % st for ordinals
\definecolor{darkblue}{rgb}{0,0,0.7} % darkblue color
\definecolor{green}{RGB}{57,181,74} % green color
\definecolor{violet}{RGB}{147,39,143} % violet color
\newcommand{\red}{\color{red}} % red command
\newcommand{\blue}{\color{blue}} % blue command
\newcommand{\orange}{\color{orange}} % orange command
\newcommand{\green}{\color{green}} % green command
\newcommand{\darkblue}{\color{darkblue}} % darkblue command
\newcommand{\defn}[1]{\textsl{\darkblue #1}} % emphasis of a definition
\newcommand{\para}[1]{\medskip\noindent\uline{\textit{#1.}}} % paragraph
\renewcommand{\topfraction}{1} % possibility to have one page of pictures
\renewcommand{\bottomfraction}{1} % possibility to have one page of pictures
\newcommand{\shade}[1]{{\color{cyan} #1}}

% marginal comments
\usepackage{todonotes}
\newcommand{\guillaume}[1]{\todo[color=orange!30]{#1 --- G.}}
\newcommand{\vincent}[1]{\todo[color=blue!30]{#1 \\ \hfill --- V.}}

% formating the part command
\makeatletter
\def\part{\@startsection{part}{1}%
\z@{.7\linespacing\@plus\linespacing}{.8\linespacing}%
{\LARGE\sffamily\centering}}
%\@addtoreset{section}{part}
\makeatother
\renewcommand{\thepart}{\Roman{part}}
%\renewcommand{\thesection}{\arabic{part}.\arabic{section}}

% formating the table of contents
\setcounter{tocdepth}{4}
\makeatletter
\def\l@section{\@tocline{1}{5pt}{0pc}{}{}}
\makeatother
\let\oldtocpart=\tocpart
\renewcommand{\tocpart}[2]{\sc\large\oldtocpart{#1}{#2}}
\let\oldtocsection=\tocsection
\renewcommand{\tocsection}[2]{\bf\oldtocsection{#1}{#2}}
\let\oldtocsubsubsection=\tocsubsubsection
\renewcommand{\tocsubsubsection}[2]{\quad\oldtocsubsubsection{#1}{#2}}

% drapeau européen
\usepackage{graphicx,calc}
\newlength\myheight
\newlength\mydepth
\settototalheight\myheight{Xygp}
\settodepth\mydepth{Xygp}
\setlength\fboxsep{0pt}
\newcommand*\inlinegraphics[1]{%
  \settototalheight\myheight{Xygp}%
  \settodepth\mydepth{Xygp}%
  \raisebox{-\mydepth}{\includegraphics[height=\myheight]{#1}}%
}

% SPECIFIC BLOCK GRAPH PERMUTREES

% COMBINATORICS

% decorations
\newcommand{\vertexSet}{V}
\newcommand{\decoration}{\delta}
\newcommandx{\Up}[2][1=v, 2=\decoration]{\smash{\overline #2}(#1)} 
\newcommandx{\Down}[2][1=v, 2=\decoration]{\smash{\underline #2}(#1)} 
\newcommandx{\up}[1][1=v]{\overline{#1}} 
\newcommandx{\down}[1][1=v]{\underline{#1}} 
\newcommandx{\updown}[1][1=v]{\overline{\underline{#1}}} 
% source and target
\newcommandx{\so}[1][1=i]{\textsc{s}(#1)} % source set
\newcommandx{\ta}[1][1=o]{\textsc{t}(#1)} % target set
% spines
\newcommandx{\spine}[1][1=S]{#1} % spine
\newcommandx{\spines}[1][1=\b{G}]{\mathcal{S}(#1)} % set of spines
\newcommandx{\spinePoset}[1][1=\b{G}]{\mathcal{P}_{#1}} % spine poset
\newcommand{\blossom}{^\text{\ding{96}}} % blossom
\newcommand{\labeling}{\lambda} % labels in blossoming trees
% nested complex
\newcommandx{\nested}[1][1=N]{#1} % nested set
\newcommand{\negNested}{\preceq} % negative nested
\newcommand{\posNested}{\succeq} % positive nested
\newcommand{\negDisjoint}{\perp} % negative disjoint
\newcommand{\posDisjoint}{\;\top\;} % positive disjoint
\newcommand{\compl}[1]{#1{}^\textsc{c}} % complement
\newcommandx{\nestedComplex}[1][1=\b{G}]{\mathcal{N}_{#1}} % nested complex
% maps
\newcommand{\spineToNested}{\mathsf{N}} % spine to nested set
\newcommand{\nestedToSpine}{\mathsf{S}} % nested set to spine
\newcommand{\partitionToSpine}{\mathsf{S}} % partition to spine
\newcommandx{\surjectionSpines}[2][1=\b{G}, 2=\b{G}']{\Phi_{#1 \to #2}} % spine to spine

% GEOMETRY

% fans
\newcommand{\primalCone}{\mathsf{C}^\star} % primal cone
\newcommand{\normalCone}{\mathsf{C}} % normal cone
\newcommandx{\braidFan}[1][1=G]{\mathcal{F}_{#1}} % braid fan
\newcommandx{\sylvesterFan}[1][1=G]{\mathcal{T}_{#1}} % sylvester fan
\newcommandx{\graphicalFan}[1][1=G]{\mathcal{G}_{#1}} % graphical fan
\newcommandx{\spineFan}[1][1=\b{G}]{\mathcal{S}_{#1}} % spine fan
% polytopes
\newcommand{\polytope}[1]{\mathsf{#1}} % font polytopes
\newcommandx{\Perm}[2][1=G,2=\weight]{\polytope{Perm}^{#2}_{#1}} % permutahedron
\newcommandx{\Asso}[2][1=G,2=\weight]{\polytope{Asso}^{#2}_{#1}} % associahedron
\newcommandx{\Spin}[2][1=\b{G},2=\weight]{\polytope{Spin}^{#2}_{#1}} % spine polytope
\newcommandx{\Zono}[2][1=G,2=\weight]{\mathsf{Zono}^{#2}_{#1}} % zonotope
\newcommandx{\point}[2][1=\spine,2=\weight]{\b{a}^{#2}_{#1}} % point
\newcommand{\weight}{\omega} % weight
\newcommandx{\boundary}[1][1=\pi]{\partial#1} % endpoints of a path
\newcommandx{\peaks}[1][1=\pi]{\wedge#1} % top of a path
\newcommandx{\valleys}[1][1=\pi]{\vee#1} % bottom of a path
\newcommandx{\Hyp}[1][1=\weight]{\mathbf{H}^{= #1}} % hyperplane
\newcommandx{\HS}[1][1=\weight]{\mathbf{H}^{\ge #1}} % halfspace
\newcommand{\monombinom}[1]{\big\{\!\begin{smallmatrix} #1 \\ 2 \end{smallmatrix}\!\big\}}

%%%%%%%%%%%%%%%%%%%%%%%%%%%%%%%%%%%%%%

\title{Spine polytopes}

\author{Guillaume Laplante-Anfossi}
\address[Guillaume Laplante-Anfossi]{Universit\'e Sorbonne Paris Nord, Laboratoire Analyse, G\'eom\'etrie et Applications, CNRS, UMR 7539, F-93430 Villetaneuse, France}
\email{laplante-anfossi@math.univ-paris13.fr}
\urladdr{\url{https://www.math.univ-paris13.fr/~laplante-anfossi/}}

\author{Vincent Pilaud}
\address[Vincent Pilaud]{CNRS \& LIX, \'Ecole Polytechnique, Palaiseau}
\email{vincent.pilaud@lix.polytechnique.fr}
\urladdr{\url{http://www.lix.polytechnique.fr/~pilaud/}}

%\date{\today}

\subjclass[2010]{Primary 52B11; Secondary 18M70} 

%\keywords{Polytopes...}

\thanks{The first author was supported by the European Union's Horizon 2020 research and innovation program under the Marie Sklodowska-Curie grant agreement No 754362 \inlinegraphics{EU.png}, by the Natural Sciences and Engineering Research Council of Canada (NSERC) and by the ANR-20-CE40-0016 Higher Algebra, Geometry and Topology. The second author is supported by the French ANR grants CAPPS~17\,CE40\,0018, and CHARMS~19\,CE40\,0017.}

%%%%%%%%%%%%%%%%%%%%%%%%%%%%%%%%%%%%%%
%%%%%%%%%%%%%%%%%%%%%%%%%%%%%%%%%%%%%%%

\begin{document}

\begin{abstract}
TBC
\end{abstract}

\maketitle

\tableofcontents

%%%%%%%%%%%%%%%%%%%%%%%%%%%%%%%%%%%%%%%
%%%%%%%%%%%%%%%%%%%%%%%%%%%%%%%%%%%%%%%

\newpage
\section*{Introduction}

Common generalization of \cite{Pilaud-signedTreeAssociahedraFPSAC, LangePilaud, PilaudPons-permutrees, Laplante-diagonalOperahedra}.
It is the maximal generalization, according to \cite{Pilaud-removahedra}.

%%%%%%%%%%%%%%%%%%%%%%%%%%%%%%%%%%%%%%%
%%%%%%%%%%%%%%%%%%%%%%%%%%%%%%%%%%%%%%%

\newpage
\section{Combinatorics: Spine Poset and Nested Complex}

%%%%%%%%%%%%%%%%%%%%%%%%%%%%%%%%%%%%%%%

\subsection{Maple trees and block graphs}
\label{subsec:MapleTreesBlockGraphs}

\begin{definition}
  A \defn{maple tree} is a tree whose vertices are properly colored (\ie no monochromatic edge) in red and blue such that all leaves are red.
  The \defn{tapping} of a maple tree consists in replacing each red vertex by a clique on its blue neighbors.
\end{definition}

\begin{definition}
  \label{def:blockgraph}
  A \defn{block graph} is a connected graph~$G$ satisfying any of the following equivalent conditions:
  \begin{enumerate}
    \item Every biconnected component of~$G$ is a clique.
    \item Any cycle in~$G$ induces a clique.
    \item The intersection of two connected subgraphs of~$G$ is a connected subgraph of~$G$.
    \item The intersection of two paths in~$G$ induces a path in~$G$.
    \item There is a unique induced path connecting every pair of vertices.
    \item The graph~$G$ is obtained by tapping a maple tree.
  \end{enumerate}
\end{definition}

Note that a block graph is obtained by tapping different maple trees. For instance, one can arbitrarily add red leaves to the blue vertices.

For convenience, in a maple tree, we label the blue vertices by digits (or numbers) and the red vertices by letters (or words).
Consequently, in a block graph, we label the vertices by digits (or numbers) and the cliques by letters (or words).
In this paper, we consider the following additional decorations.

\begin{definition}
  A \defn{decoration} of a set~$\vertexSet$ is a map~$\decoration : \vertexSet \to \{\circ, \up[\circ], \down[\circ], \updown[\circ]\}$.
  For~$U \subseteq \vertexSet$, we define
  \[
    \Up[U] \eqdef \bigset{u \in U}{\decoration(u) \in \{\up[\circ], \updown[\circ]\}}
    \qquad\text{and}\qquad
    \Down[U] \eqdef \bigset{u \in U}{\decoration(u) \in \{\down[\circ], \updown[\circ]\}}.
  \]
  A \defn{decorated maple tree} is a pair~$\b{M} \eqdef (M,\decoration)$ where~$M$ is a maple tree and $\decoration$ is a decoration on the set of blue vertices of~$M$.
  A \defn{decorated block graph} is a pair~$\b{G} \eqdef (G,\decoration)$ where~$G$ is a block graph and $\decoration$ is a decoration on the vertex set of~$G$.
\end{definition}

\begin{example}
  \label{exm:specialGraphs}
  Our results will be illustrated with the following special cases:
  \begin{enumerate}[(i)]
    \item special block graphs: complete graphs, trees, or paths.
    \item special decorations: undecorated ($\Down[\vertexSet] = \Up[\vertexSet] = \varnothing$), down decorated ($\Down[\vertexSet] = \vertexSet$ and~${\Up[\vertexSet] = \varnothing}$), up-down decorated ($\Down[\vertexSet] \sqcup \Up[\vertexSet] = \vertexSet$), and fully decorated ($\Down[\vertexSet] = \Up[\vertexSet] = \vertexSet$).
    \item toy example: the maple tree and block graph of \cref{fig:mapleBlock}.
    \item low-dimensional examples: the four decorated tripod graphs of \cref{fig:tripods}.
  \end{enumerate}
\end{example}

\begin{figure}
\resizebox{\linewidth}{!}{
\begin{tikzpicture}

\node (1) at (-4, -19) [circle,draw=none,minimum size=4mm,inner sep=0.1mm]{$\blue 1$};
\node (4) at (-4, -21) [circle,draw=none,minimum size=4mm,inner sep=0.1mm]{$\blue 4$};
\node (5) at (-2, -21) [circle,draw=none,minimum size=4mm,inner sep=0.1mm]{$\blue \up[5]$};
\node (7) at (-4, -23) [circle,draw=none,minimum size=4mm,inner sep=0.1mm]{$\blue \down[7]$};
\node (8) at (0, -23) [circle,draw=none,minimum size=4mm,inner sep=0.1mm]{$\blue 8$};
\node (9) at (2, -23) [circle,draw=none,minimum size=4mm,inner sep=0.1mm]{$\blue \up[9]$};
\node (6) at (2, -21) [circle,draw=none,minimum size=4mm,inner sep=0.1mm]{$\blue 6$};
\node (2) at (0, -19) [circle,draw=none,minimum size=4mm,inner sep=0.1mm]{$\blue \updown[2]$};
\node (3) at (2, -19) [circle,draw=none,minimum size=4mm,inner sep=0.1mm]{$\blue 3$};

\node (f) at (0, -21) [circle,draw=none,minimum size=4mm,inner sep=0.1mm]{$\red f$};
\node (d) at (-3, -20) [circle,draw=none,minimum size=4mm,inner sep=0.1mm]{$\red d$};
\node (a) at (-5, -19) [circle,draw=none,minimum size=4mm,inner sep=0.1mm]{$\red a$};
\node (e) at (-5, -21) [circle,draw=none,minimum size=4mm,inner sep=0.1mm]{$\red e$};
\node (i) at (-5, -23) [circle,draw=none,minimum size=4mm,inner sep=0.1mm]{$\red i$};
\node (h) at (-3, -22) [circle,draw=none,minimum size=4mm,inner sep=0.1mm]{$\red h$};
\node (j) at (1, -23) [circle,draw=none,minimum size=4mm,inner sep=0.1mm]{$\red j$};
\node (k) at (3, -23) [circle,draw=none,minimum size=4mm,inner sep=0.1mm]{$\red k$};
\node (g) at (3, -21) [circle,draw=none,minimum size=4mm,inner sep=0.1mm]{$\red g$};
\node (b) at (1, -19) [circle,draw=none,minimum size=4mm,inner sep=0.1mm]{$\red b$};
\node (c) at (3, -19) [circle,draw=none,minimum size=4mm,inner sep=0.1mm]{$\red c$};
  
\draw[-] (a)--(1)--(d)--(4)--(e); 
\draw[-] (d)--(5)--(h)--(7)--(i); 
\draw[-] (5)--(f)--(6)--(g); 
\draw[-] (c)--(3)--(b)--(2)--(f)--(8)--(j)--(9)--(k); 

\end{tikzpicture}
\quad
\quad
\quad
\begin{tikzpicture}

  \node (1) at (-4, -19) [circle,draw=none,minimum size=4mm,inner sep=0.1mm]{$\blue 1$};
  \node (4) at (-4, -21) [circle,draw=none,minimum size=4mm,inner sep=0.1mm]{$\blue 4$};
  \node (5) at (-2, -21) [circle,draw=none,minimum size=4mm,inner sep=0.1mm]{$\blue \up[5]$};
  \node (7) at (-4, -23) [circle,draw=none,minimum size=4mm,inner sep=0.1mm]{$\blue \down[7]$};
  \node (8) at (0, -23) [circle,draw=none,minimum size=4mm,inner sep=0.1mm]{$\blue 8$};
  \node (9) at (2, -23) [circle,draw=none,minimum size=4mm,inner sep=0.1mm]{$\blue \up[9]$};
  \node (6) at (2, -21) [circle,draw=none,minimum size=4mm,inner sep=0.1mm]{$\blue 6$};
  \node (2) at (0, -19) [circle,draw=none,minimum size=4mm,inner sep=0.1mm]{$\blue \updown[2]$};
  \node (3) at (2, -19) [circle,draw=none,minimum size=4mm,inner sep=0.1mm]{$\blue 3$};

  \draw[-] (1)--(4)--(5)--(1); 
  \draw[-] (2)--(6)--(8)--(5); 
  \draw[-] (7)--(5)--(2)--(3); 
  \draw[-] (2)--(8)--(9); 
  \draw[-] (5)--(6); 

\end{tikzpicture}}
\caption{A decorated maple tree and its associated block graph}
\label{fig:mapleBlock}
\end{figure} 

%%%%%%%%%%%%%%%%%%%%%%%%%%%%%%%%%%%%%%%

\subsection{Spines}
\label{subsec:spines}

\begin{definition}
  \label{def:sourceTargetSets}
%  Consider an arc~$\gamma$ in a directed tree~$\spine$ whose nodes are sets.
%  The \defn{source set}~$\so[\gamma]$ (resp.~\defn{target set}~$\ta[\gamma]$) of~$\gamma$ is the union of the nodes in the connected component of~$\spine \ssm \{\gamma\}$ containing the source (resp.~target)~of~$\gamma$.
  In a directed tree~$\spine$ whose nodes are sets,
  \begin{itemize}
    \item the \defn{source set}~$\so[\gamma]$ (resp.~\defn{target set}~$\ta[\gamma]$) of an arc~$\gamma$ is the union of the nodes in the connected component of $\spine \ssm \{\gamma\}$ containing the source (resp.~the target) of~$\gamma$,
    \item the \defn{source set}~$\so[X]$ (resp.~\defn{target set}~$\ta[X]$) of a node~$X$ is the union of the source sets~$\so[\alpha]$ of all incoming arcs~$\alpha$ (resp.~of the target sets~$\ta[\beta]$ of all outgoing arcs~$\beta$) of~$X$~in~$\spine$.
  \end{itemize}
\end{definition}

%In this paper, we consider directed trees whose nodes are labeled by sets.
%With a slight abuse of notation, we still write~$\so[\gamma]$ (resp.~$\ta[\gamma]$) for the union of the labels in the source (resp.~target) set of an arc~$\gamma$.

\begin{definition}
  \label{def:spine}
  A \defn{spine} on a decorated block graph $\b{G} \eqdef (G,\decoration)$ is a directed tree $\spine$ such that
  \begin{enumerate}
    \item the nodes of~$\spine$ form a partition of the vertex set~$\vertexSet$ of~$G$, and 
    \item at each node~$X$ of~$\spine$, the source sets $\so[\alpha]$ of the incoming arcs $\alpha$ are contained in distinct connected components of $G \ssm \Down[X]$, and the target sets $\ta[\beta]$ of the outgoing arcs $\beta$ are contained in distinct connected components of $G \ssm \Up[X]$.
  \end{enumerate}
%  We denote by $\spines$ the set of spines on~$\b{G}$.
%  \vincent{not sure we really need this notation}
\end{definition}

\begin{example}
  \label{exm:spines}
  For instance, the spines correspond to:
  \vincent{todo: other ideas?}
  \begin{enumerate}[(i)]
    \item the \defn{ordered partitions} of~$\vertexSet$ when $\b{G}$ is complete or undecorated.
    \item the \defn{$G$-trees} studied in~\cite{Postnikov} when~$\b{G}$ is down decorated (in particular the classical \defn{Schr\"oder trees} when~$\b{G}$ is a down decorated path).
    \item the \defn{$G$-ordered partitions}~\cite{Pilaud-acyclicReorientationLattices} (\ie pairs~$(p,o)$ where~$p$ is a partition of~$\vertexSet$ into connected subgraphs of~$G$, and~$o$ is an acyclic orientation of the quotient graph~$G/p$) when~$\b{G}$ is fully decorated.
    \item the \defn{Schr\"oder permutrees} of~\cite{PilaudPons-permutrees} when~$G$ is a path.
  \end{enumerate}
  \vincent{give two examples of spines on the block graph of \cref{fig:mapleBlock}, one maximal and one not.}
\end{example}

\vincent{to clean here}
%\begin{proposition}
%  \label{prop:spines}
%  Let $\b{G}$ be a decorated block graph, let $\spine$ be a spine on $\b{G}$, and let~$\gamma$ be an arc of~$\spine$ joining a node~$X$ to a node~$Y$.
%  Then
%  \begin{enumerate}[(i)]
%    \item $\so[\gamma]$ and~$\ta[\gamma]$ form a non-trivial partition of the vertex set of~$G$, with $\so[\gamma]$ contained in a connected component of~$G \ssm \Down[{\ta[\gamma]}]$ and $\ta[\gamma]$ contained in a connected component of~$G \ssm \Up[{\so[\gamma]}]$,
%    \item 
%  \end{enumerate}
%\end{proposition}

%%%%%%%%%%%%%%%%%%%%%%%%%%%%%%%%%%%%%%%

\subsection{Spine poset}
\label{subsec:spinePoset}

We now define two natural operations on spines: arc contraction and node splitting.
These operations are illustrated in \cref{fig:contractionSplitting}.

\begin{definition}
  \label{def:arccontraction}
  Let~$\gamma$ be an arc of a spine~$\spine$ on $\b{G}$, joining a node~$X$ to a node~$Y$.
  The \defn{contraction} of~$\gamma$ in~$\spine$ is the directed tree obtained from~$\spine$ by replacing the nodes~$X$ and~$Y$ by a single node~$Z \eqdef X \cup Y$ and attaching to~$Z$ all arcs incident to either~$X$~or~$Y$.
  See \cref{fig:contractionSplitting}.
\end{definition}

\begin{figure}[h!]
\centering

\begin{tikzpicture}[scale=1.6]
    
\node (N0) [circle,draw=none,minimum size=4mm,inner sep=0.1mm] at (-0.5,-0.5) {\small $X$};
\node (N1) [circle,draw=none,minimum size=4mm,inner sep=0.1mm] at (0.5,0.5) {\small $Y$};

\node (o1) [circle,draw=none,minimum size=4mm,inner sep=0.1mm] at (-1.2,0.6) {\small $\beta_1$};
\node (od) [circle,draw=none,minimum size=4mm,inner sep=0.1mm] at (-0.8,0.6) {\small $\dots$};
\node (oj) [circle,draw=none,minimum size=4mm,inner sep=0.1mm] at (-0.3,0.6) {\small $\beta_k$};

\node (oj1) [circle,draw=none,minimum size=4mm,inner sep=0.1mm] at (0,1.62) {\small $\beta_{k+1}$};
\node (ojd) [circle,draw=none,minimum size=4mm,inner sep=0.1mm] at (0.5,1.6) {\small $\dots$};
\node (ol) [circle,draw=none,minimum size=4mm,inner sep=0.1mm] at (1,1.6) {\small $\beta_\ell$};

\node (i1) [circle,draw=none,minimum size=4mm,inner sep=0.1mm] at (-1,-1.6) {\small $\alpha_1$};
\node (id) [circle,draw=none,minimum size=4mm,inner sep=0.1mm] at (-0.5,-1.6) {\small $\dots$};
\node (im) [circle,draw=none,minimum size=4mm,inner sep=0.1mm] at (0,-1.6) {\small $\alpha_i$};

\node (im1) [circle,draw=none,minimum size=4mm,inner sep=0.1mm] at (0.3,-0.62) {\small $\alpha_{i+1}$};
\node (imd) [circle,draw=none,minimum size=4mm,inner sep=0.1mm] at (0.8,-0.6) {\small $\dots$};
\node (ik) [circle,draw=none,minimum size=4mm,inner sep=0.1mm] at (1.2,-0.6) {\small $\alpha_j$};

\draw[->] (N0)--(o1); 
\draw[->] (N0)--(od);
\draw[->] (N0)--(oj);
  
\draw[->] (N0)--(N1) node[midway,left] {$\gamma$} ; 
\draw[->] (N1)--(oj1); 
\draw[->] (N1)--(ojd);
\draw[->] (N1)--(ol);

\draw[->] (i1)--(N0);
\draw[->] (id)--(N0);
\draw[->] (im)--(N0); 

\draw[->] (im1)--(N1); 
\draw[->] (imd)--(N1);
\draw[->] (ik)--(N1);

\end{tikzpicture}
\quad 
\quad 
\resizebox{0.1\linewidth}{!}{
  \raisebox{5.5em}{
  \begin{tikzpicture}
    \draw[->] (0,0)--(1,0); 
    \draw[->] (1,-0.2)--(0,-0.2); 
  \end{tikzpicture}}}
\quad 
\quad
\raisebox{1.5em}{
\begin{tikzpicture}[scale=1.2]
    
  \node (N1) [circle,draw=none,minimum size=4mm,inner sep=0.1mm] at (0,0) {\small $X \cup Y$};
  
  
  \node (oj1) [circle,draw=none,minimum size=4mm,inner sep=0.1mm] at (-0.6,1.5) {\small $\beta_1$};
  \node (ojd) [circle,draw=none,minimum size=4mm,inner sep=0.1mm] at (0,1.5) {\small $\dots$};
  \node (ol) [circle,draw=none,minimum size=4mm,inner sep=0.1mm] at (0.6,1.5) {\small $\beta_\ell$};
  
  
  \node (im1) [circle,draw=none,minimum size=4mm,inner sep=0.1mm] at (-0.6,-1.5) {\small $\alpha_1$};
  \node (imd) [circle,draw=none,minimum size=4mm,inner sep=0.1mm] at (0,-1.5) {\small $\dots$};
  \node (ik) [circle,draw=none,minimum size=4mm,inner sep=0.1mm] at (0.6,-1.5) {\small $\alpha_j$};
  
  
  \draw[->] (N1)--(oj1); 
  \draw[->] (N1)--(ojd);
  \draw[->] (N1)--(ol);
  
  \draw[->] (im1)--(N1); 
  \draw[->] (imd)--(N1);
  \draw[->] (ik)--(N1);
  

\end{tikzpicture}}
\caption{Contraction and splitting on spines.}
\label{fig:contractionSplitting}
\end{figure} 

\begin{proposition}
  \label{prop:arccontraction}
  The contraction of any arc in any spine on~$\b{G}$ is a spine on~$\b{G}$.
\end{proposition}

\begin{proof} 
  Let $\spine'$ be a directed tree obtained by contracting an arc $\gamma$ joining~$X$ to~$Y$ in a spine $\spine$ on~$\b{G}$.
  We clearly just need to prove the local condition of \cref{def:spine} around the node~$Z \eqdef X \cup Y$ of~$\spine'$.
  We give the argument for the incoming arcs; the argument for the outgoing arcs is symmetric. 

  To prove that the source set~$\so[\alpha]$ of each incoming arc~$\alpha$ of~$Z$ is contained in a connected component of~$G \ssm \Down[Z]$, we distinguish two cases:
  \begin{itemize}
    \item If~$\alpha$ is an incoming arc of~$X$ in~$\spine$, then its source set~$\so[\alpha]$ is contained in a connected component $C_1$ of $G \ssm \Down[X]$ and also in a connected component $C_2$ of $G \ssm \Down[Y]$. Thus, it is in $C_1 \cap C_2$. By \cref{def:blockgraph}, the subgraph induced by the vertices of $C_1 \cap C_2$ is a connected component of $(G \ssm \Down[X]) \cap (G \ssm \Down[Y]) = G \ssm \Down[Z]$.
    \item If~$\alpha$ is an incoming arc of~$Y$ in~$\spine$, then its source set $\so[\alpha]$ is in a connected component of $G \ssm \Down[Y]$ that does not contain $X$, hence in a connected component of $G \ssm \Down[Z]$.
  \end{itemize}

  To prove that the source sets~$\so[\alpha]$ and~$\so[\alpha']$ of two distinct incoming arcs~$\alpha$ and~$\alpha'$ of~$Z$ are contained in two distinct connected components of~$G \ssm \Down[Z]$, we distinguish two cases:
  \begin{itemize}
    \item If~$\alpha$ and~$\alpha'$ are both incoming arcs of~$X$ (resp.~of~$Y$) in~$\spine$, then their source sets~$\so[\alpha]$ and~$\so[\alpha']$ live in two distinct connected components of $G \ssm \Down[X]$ (resp.~of~$G \ssm \Down[Y]$), hence in two distinct connected components of $G\ssm \Down[Z]$.
    \item If~$\alpha$ is an incoming arc of~$X$ in~$\spine$ while~$\alpha'$ is an incoming arc of~$Y$ in~$\spine$, then their source sets~$\so[\alpha] \subset \so[\gamma]$ and~$\so[\alpha']$ live in two distinct connected components of $G \ssm \Down[Y]$, hence in two distinct connected components of $G \ssm \Down[Z]$.
    \qedhere
  \end{itemize}
\end{proof}

To define the reverse operation of node splitting, we first need the following notion.

\begin{definition}
  \label{def:splittable}
  Let~$Z$ be a subset of vertices of~$G$.
  A partition~$X \sqcup Y = Z$ is \defn{$\b{G}$-splittable} if~$X \ne \varnothing$ is contained in a connected component of~$G \ssm \Down[Y]$ and $Y \ne \varnothing$ is contained in a connected component of~$G \ssm \Up[X]$.
\end{definition}

\begin{proposition}
  \label{prop:splittablePartitions}
  Any subset $Z$ of vertices of~$G$ with $|Z| \ge 2$ admits a $\b{G}$-splittable partition.
\end{proposition}

\begin{proof} 
  Consider a spanning tree~$T$ of~$G$, and an edge~$e$ of~$T$ such that both connected components~$C$ and~$D$ of~$T \ssm \{e\}$ intersect~$Z$.
  Let~$X \eqdef Z \cap C$ and~$Y \eqdef Z \cap D$.
  Then~$X \ne \varnothing$ is contained in a connected component of~$T \ssm D$, thus in a connected component of~$G \ssm \Down[Y]$ (since~$T \subseteq G$ and~$\Down[Y] \subseteq D$).
  Similarly $Y \ne \varnothing$ is contained in a connected component of~$T \ssm C$, thus in a connected component of~$G \ssm \Up[X]$.
\end{proof}

\begin{definition}
  \label{def:nodeSplitting} 
  %Let $\spine$ be a spine on $\b{G}$, let~$Z$ be a node of~$\spine$, and let~$X \sqcup Y = Z$ be a splittable partition of~$Z$.
  Let~$X \sqcup Y = Z$ be a splittable partition of a node~$Z$ of a spine~$\spine$ on~$\b{G}$.
  The \defn{splitting} of~$X \sqcup Y$ in~$\spine$ is the directed tree obtained from~$\spine$ by replacing the node~$Z$ by an arc~$\gamma$ connecting a node~$X$ to a node~$Y$,~and
  \begin{itemize}
    \item connecting each incoming arc~$\alpha$ of~$Z$ to~$X$ if its source set~$\so[\alpha]$ is contained in a connected component of $G \ssm \Down[Z]$ adjacent to $\Down[X]$ and to~$Y$ otherwise, 
    \item connecting each outgoing arc~$\beta$ of~$Z$ to~$Y$ if its target set~$\ta[\beta]$ is contained in a connected component of $G \ssm \Up[Z]$ adjacent to $\Up[Y]$ and to~$X$ otherwise.
  \end{itemize}
  See \cref{fig:contractionSplitting}.
\end{definition}

\begin{proposition}
  \label{prop:nodeSplitting} 
  The splitting of any splittable partition in any spine on~$\b{G}$ is a spine on~$\b{G}$.
\end{proposition}

%\begin{remark}
%  The following proof is "formal", i.e. it doesn't use the fact that $G$ is a block graph. 
%\end{remark}

\begin{proof}
  Let~$\spine'$ be a directed tree obtained by splitting a splittable partition~$X \sqcup Y = Z$ of a node~$Z$ in a spine $\spine$ on~$\b{G}$.
  We clearly just need to prove the local condition of \cref{def:spine} around the nodes~$X$ and~$Y$ of~$\spine'$.
  We give the argument for~$X$; the argument for~$Y$ is symmetric.
  
  We start with the incoming arcs of~$X$.
  First, each incoming arc~$\alpha$ of~$X$ is an incoming arc of~$Z$, hence~$\so[\alpha]$ is contained in a connected component of~$G \ssm \Down[Z]$, hence in a connected component of~$G \ssm \Down[X]$.
  Moreover, any two distinct incoming arcs~$\alpha$ and~$\alpha'$ of~$X$ are distinct incoming arcs of~$Z$. Thus~$\so[\alpha]$ and~$\so[\alpha']$ are contained in distinct connected components of~$G \ssm \Down[Z]$, hence in distinct connected components of~$G \ssm \Down[X]$, since~$X$ is contained in a connected component of~$G \ssm \Down[Y]$.

  We now consider the outgoing arcs of~$X$.
  To prove that the target set of each outgoing arc~$\beta$ of~$X$ is contained in a connected component of~$G \ssm \Up[X]$, we distinguish two cases:
  \begin{itemize}
    \item If~$\beta \ne \gamma$, then~$\beta$ is an outgoing arc of~$Z$, hence~$\ta[\beta]$ is contained in a connected component of~$G \ssm \Up[Z]$, hence in a connected component of~$G \ssm \Up[X]$.
    \item If~$\beta = \gamma$, then~$\ta[\gamma]$ is the union of~$Y$ with~$\so[\alpha']$ for all incoming arcs~$\alpha'$ of~$Y$ and~$\ta[\beta']$ for all outgoing arcs~$\beta'$ of~$Y$. We next prove that all these sets are contained in the same connected component of~$G \ssm \Up[X]$.
    \begin{itemize}
      \item By \cref{def:nodeSplitting}, $Y$ is contained in a connected component~$C$ of~$G \ssm \Up[X]$. 
      \item Consider an incoming arc~$\alpha'$ of~$Y$. The set~$\so[\alpha']$ is contained in a connected component~$D$ of~$G \ssm \Down[Z]$, and in a connected component~$E$ of~$G \ssm \Down[Y]$, with~$D \subseteq E$. By \cref{def:nodeSplitting}, $D$ is not adjacent to~$\Down[X]$. By \cref{def:splittable}, $E$ is thus disjoint from~$X$. This implies on the one hand that~$D$ is disjoint from~$\Up[X]$, thus connected in~$G \ssm \Up[X]$, and on the other hand that~$D$ is adjacent to~$\Down[Y]$, hence contained in~$C$. 
      \item Consider an outgoing arc~$\beta'$ of~$Y$. By \cref{def:nodeSplitting}, the set $\ta[\beta']$ is contained in a connected component of~$G \ssm \Up[Z]$ adjacent to~$Y$, hence $\ta[\beta']$ is contained in~$C$.
    \end{itemize}
  \end{itemize}
  To prove that the target sets of two distinct outgoing arcs~$\beta$ and~$\beta'$ of~$X$ are contained in two distinct connected components of~$G \ssm \Up[X]$, we distinguish two cases:
  \begin{itemize}
  \item Assume that~$\beta$ and~$\beta'$ are both distinct from~$\gamma$. Let~$C$ be the connected component of~$G \ssm \Up[X]$ containing~$Y$, and let~$D$ and~$D'$ (resp.~$E$ and~$E'$) the connected components of $G \ssm \Up[Z]$ (resp.~$G \ssm \Up[X]$) containing~$\ta[\beta]$ and $\ta[\beta']$. By \cref{def:nodeSplitting}, $D$ and~$D'$ are not adjacent to~$\Up[Y]$. By \cref{def:splittable}, $E$ and~$E'$ are thus disjoint from~$Y$. This implies that~$E = D$ and~$E' = D'$ are distinct.
  \item Assume that~$\beta' = \gamma$. As before, the connected component of~$G \ssm \Up[X]$ containing~$\ta[\beta]$ is disjoint from~$Y$, thus is distinct from that containing~$\ta[\gamma]$.
  \qedhere
  \end{itemize}
\end{proof}

\begin{proposition}
  \label{prop:contractionSplitting}
  Arc contraction and node splitting are inverse operations.
\end{proposition}

\begin{proof}
  It is clear that if~$\spine'$ is the splitting of~$X \sqcup Y$ in~$\spine$, then~$\spine$ is the contraction of the arc joining~$X$ to~$Y$ in~$\spine'$.
  Conversely, if~$\spine$ is the contraction of an arc joining~$X$ to~$Y$ in~$\spine'$, then the partition~$X \sqcup Y$ is splittable by \cref{def:spine}, and~$\spine'$ is the splitting of~$X \sqcup Y$~in~$\spine$.
\end{proof}

\begin{definition}
  The \defn{spine poset}~$\spinePoset$ is the poset on spines on~$\b{G}$ defined by~$\spine \le \spine'$ if and only if~$\spine$ is obtained from~$\spine'$ by arc contractions, or equivalently~$\spine'$ is obtained from~$\spine$ by node~splittings.
\end{definition}

\begin{corollary} 
  The spine poset is graded by the number of nodes.
  Hence, the cover relations of the spine poset are precisely given by arc contractions, or equivalently node splittings.
  Moreover,
  \begin{itemize}
    \item the unique rank $0$ (\ie minimal) element is the spine with a single node, 
    \item the rank~$1$ elements are the spines with precisely two nodes, 
    \item the corank~$1$ elements are the spines where the nodes are all singletons except one pair,
    \item the corank~$0$ (\ie maximal) elements are the spines where the nodes are all singletons.
  \end{itemize}
\end{corollary}

\begin{remark}
  In the spine poset:
  \begin{itemize}
    \item a spine with~$a$ arcs covers~$a$ spines, and is larger than precisely $\binom{a}{b}$ spines with~$b$ arcs,
    \item a spine with~$k$ nodes of cardinality~$n_1, \dots, n_k$ is covered by at least~$2^{|\set{i \in [k]}{n_i \ge 2}|}$ and at most~$\prod_{i \in [k]} 2^{n_i}-2$ spines. In particular, a corank~$1$ spine is covered by precisely $2$ maximal~spines.
  \end{itemize}
\end{remark}

\begin{remark}
  All the results in this section hold if and only if $G$ is a block graph. 
  \vincent{Not the right place...}
\end{remark}

\begin{corollary}
  \label{coro:splittablePartitions}
  Let~$\gamma$ be an arc of a spine~$\spine$ on~$\b{G}$, joining a node~$X$ to a node~$Y$.
  Then~$\so[\gamma]$ and~$\ta[\gamma]$ form a $\b{G}$-splittable partition of the vertex set of~$G$.
\end{corollary}

\begin{proof}
  Consider the rank~$1$ spine~$\spine'$ on~$\b{G}$ obtained by contracting all arcs in~$\spine$ except~$\gamma$. 
  By \cref{prop:contractionSplitting}, $\spine$ is obtained from the rank~$0$ spine by a node splitting. 
  Hence, $\so[\gamma]$ and~$\ta[\gamma]$ form a $\b{G}$-splittable partition of the vertex set of~$G$.
\end{proof}

%%%%%%%%%%%%%%%%%%%%%%%%%%%%%%%%%%%%%%%

\subsection{Nested complex}

\vincent{Do we want to say that the spine complex is a simplicial poset, which we want to prove is flag?}
% We denote by~$\compl{B} \eqdef \vertexSet \ssm B$ the complement of a subset~$B$ of~$\vertexSet$.

\begin{definition}
  \label{def:block}
  A \defn{block} of~$\b{G}$ is a subset~$B$ of vertices of~$G$ such that~$B$ and its complement~$\vertexSet \ssm B$ forms a $\b{G}$-splittable partition of the vertex set~$\vertexSet$ (see \cref{def:splittable}).
%  A \defn{block} of~$\b{G}$ is a subset~$B$ of~$\vertexSet$ such that $B \ne \varnothing$ is contained in a connected component of~$G \ssm \Down[\compl{B}]$ and $\compl{B} \ne \varnothing$ is contained in a connected component of~$G \ssm \Up[B]$.
%  In other words, $B$ and its complement~$\compl{B}$ forms a splittable partition of~$\vertexSet$ (see \cref{def:splittable}).
\end{definition}

\begin{definition}
  \label{def:compatibleBlocks}
%  A \defn{block} of~$\b{G}$ is a subset~$B$ of vertices of~$G$ such that~$B$ and its complement~$\vertexSet \ssm B$ forms a $\b{G}$-splittable partition of the vertex set of~$G$ (see \cref{def:splittable}).
  %~$B \ne \varnothing$ is contained in a connected component of~$G \ssm \compl{B}$ and $\compl{B} \ne \varnothing$ is contained in a connected component of~$G \ssm B$.
%  Two blocks~$B$ and~$B'$ are \defn{compatible} if any of the following conditions hold:
%  \begin{itemize}
%    \item $B \subseteq B'$,
%    \item $B \supseteq B'$,
%    \item $B \cap B' = \varnothing$ and $B \cup B'$ is not a block,
%    \item $\compl{B} \cup \compl{B'} = \varnothing$ and~$B \cap B'$ is not a block.
%  \end{itemize}
  Let~$B$ and~$B'$ be two blocks of~$\b{G}$.
  We write:
  \begin{itemize}
    \item $B \negDisjoint B'$ if $B \cap B' = \varnothing$ and $B \cup B'$ is not a block of~$\b{G}$,
    \item $B \posDisjoint B'$ if $B \cup B' = \vertexSet$ and~$B \cap B'$ is not a block of~$\b{G}$.
  \end{itemize}
  We say that~$B$ and~$B'$ are \defn{compatible} if $B \subseteq B'$, or $B \supseteq B'$, or $B \negDisjoint B'$ or $B \posDisjoint B'$.
\end{definition}

\begin{definition}
  \label{def:spineComplex}
  A \defn{nested set} on~$\b{G}$ is a set of pairwise compatible blocks of~$\b{G}$.
  The \defn{nested complex}~$\nestedComplex$ is the simplicial complex of nested sets on~$\b{G}$.
  In other words, it is the clique complex of the compatibility relation on blocks of~$\b{G}$.
\end{definition}

%\begin{lemma}
%  \label{lem:arcBlock}
%  The source set~$\so[\gamma]$ of an arc~$\gamma$ of a spine~$\spine$ on~$\b{G}$ is a block of~$\b{G}$.
%\end{lemma}
%
%\begin{proof}
%  Let~$\spine'$ be the rank~$1$ spine on~$\b{G}$ obtained by contracting all arcs in~$\spine$ except~$\gamma$.
%  By \cref{prop:contractionSplitting}, $\spine$ is obtained from the rank~$0$ spine by a node splitting.
%  This implies that the partition~$\so[\gamma] \sqcup \ta[\gamma]$ is splittable.
%\end{proof}
%
%\begin{lemma}
%  \label{lem:compatibleArcBlocks}
%  For two arcs~$\gamma$ and~$\gamma'$ of a spine~$\spine$ on~$\b{G}$, the blocks~$\so[\gamma]$ and~$\so[\gamma']$ of~$\b{G}$ are compatible.
%\end{lemma}
%
%\begin{proof}
%  Since the spine~$\spine$ is a tree, there is a unique path~$\pi$ joining~$\gamma$ to~$\gamma'$ in~$\spine$.
%  If $\pi$ connects the target of~$\gamma$ to the source of~$\gamma'$, then~$\so[\gamma] \subseteq \so[\gamma']$, and \viceversa.
%  If $\pi$ connects the sources of~$\gamma$ and~$\gamma'$, then $\so[\gamma] \negDisjoint \so[\gamma]'$.
%  Indeed, there is at least one node~$X$ of~$\spine$ where~$\pi$ has two incoming arcs~$\alpha, \alpha'$.
%  By \cref{def:spine}, $\so[\alpha]$ and~$\so[\alpha']$ belong to two distinct connected components of~$G \ssm \Down[X]$.
%  Hence, we obtain that~$\so[\gamma] \subseteq \so[\alpha]$ and~$\so[\gamma] \subseteq \so[\alpha']$ belong to distinct connected components of~$G \ssm \Down[X]$ hence of~$G \ssm \Down[\compl{(\so[\gamma] \cup \so[\alpha'])}]$.
%  Similarly, if $\pi$ connects the targets of~$\gamma$ and~$\gamma'$, then $\so[\gamma] \posDisjoint \so[\gamma]'$.
%  See \cref{fig:sourceSetsCompatible}.
%  \vincent{borrow picture...}
%\end{proof}
%
%\begin{corollary}
%  Any spine on~$\b{G}$ defines a nested set on~$\b{G}$.
%\end{corollary}

\begin{proposition}
  \label{prop:spineToNested}
  The map~$\spineToNested$ defined by~$\spineToNested(\spine) \eqdef \set{\so[\gamma]}{\gamma \text{ arc of } \spine}$ is a poset isomorphism from the spine poset~$\spinePoset$ to the face poset of the nested complex~$\nestedComplex$.
\end{proposition}

\begin{proof}
  We prove that the map~$\spineToNested$ is well-defined, injective, surjective, and order-preserving.
  
  \para{Well-defined}
  Observe that for any spine~$\spine$ on~$\b{G}$:
  \begin{itemize}
    \item For any arc~$\gamma$ of~$\spine$, the source set~$\so[\gamma]$ is a block of~$\b{G}$ by \cref{coro:splittablePartitions}. 
%    Indeed, consider the rank~$1$ spine~$\spine'$ on~$\b{G}$ obtained by contracting all arcs in~$\spine$ except~$\gamma$. 
%    By \cref{prop:contractionSplitting}, $\spine$ is obtained from the rank~$0$ spine by a node splitting. 
%    Hence, the partition~$\so[\gamma] \sqcup \ta[\gamma]$ is splittable.
    \item For any two arcs~$\gamma$ and~$\gamma'$ of~$\spine$, the blocks~$\so[\gamma]$ and~$\so[\gamma']$ of~$\b{G}$ are compatible. 
    Indeed, since the spine~$\spine$ is a tree, there is a unique path~$\pi$ joining~$\gamma$ to~$\gamma'$ in~$\spine$.
    As illustrated in \cref{fig:sourceSetsCompatible}, we observe that:
    \vincent{borrow picture...}
    \begin{itemize}
      \item If $\pi$ connects the target of~$\gamma$ to the source of~$\gamma'$, then~$\so[\gamma] \subseteq \so[\gamma']$.
      \item If $\pi$ connects the source of~$\gamma$ to the target of~$\gamma'$, then~$\so[\gamma] \supseteq \so[\gamma']$.
      \item If $\pi$ connects the sources of~$\gamma$ and~$\gamma'$, then $\so[\gamma] \negDisjoint \so[\gamma]'$.
      Indeed, there is at least one node~$X$ of~$\spine$ where~$\pi$ has two incoming arcs~$\alpha, \alpha'$.
      By \cref{def:spine}, $\so[\alpha]$ and~$\so[\alpha']$ belong to two distinct connected components of~$G \ssm \Down[X]$.
      Hence, we obtain that~$\so[\gamma] \subseteq \so[\alpha]$ and~$\so[\gamma] \subseteq \so[\alpha']$ belong to distinct connected components of~$G \ssm \Down[X]$ hence of~$G \ssm \Down[\compl{(\so[\gamma] \cup \so[\alpha'])}]$.
      \item Similarly, if $\pi$ connects the targets of~$\gamma$ and~$\gamma'$, then $\so[\gamma] \posDisjoint \so[\gamma]'$.
    \end{itemize}
  \end{itemize}

  \para{Injective}
  To see the injectivity of~$\spineToNested$, we prove that we can reconstruct a spine~$\spine$ from the source sets of its arcs.
  First the the nodes of~$\spine$ are the equivalence classes under the relation~$v \equiv w$ if there is no arc~$\gamma$ of~$\spine$ such that~$|\{v,w\} \cap \so[\gamma]| = 1$.
  Second, the leaves of~$\spine$ correspond to the source sets containing either only one or all but one node of~$\spine$. 
  We can then delete one leaf~$\ell$ of~$\spine$, and reconstruct by induction the tree~$\spine \ssm \{\ell\}$. 
  Finally the only possible node of~$\spine \ssm \{\ell\}$ to which the leaf~$\ell$ can be glued is the unique node which is in the intersection of all source sets~$B$ of~$\spine \ssm \{\ell\}$ such that~$B \cup \ell$ is a source set of~$\spine$, but not in the union of all source sets~$B$ of~$\spine \ssm \{\ell\}$ such that~$B \cup \ell$ is not a source set of~$\spine$.

  \para{Surjective}
  To see the surjectivity of~$\spineToNested$, we prove that for any nested set~$\nested$ on~$\b{G}$,
\begin{enumerate}[(a)]
\item there exists a spine~$\spine$ on~$\b{G}$ such that~$\spineToNested(\spine) = \nested$, and
\item for any block~$B$ of~$\b{G}$ compatible with all blocks of~$\nested$, there exists a node~$X$ of~$\spine$ such that the partition~$X = (X \cap B) \sqcup (X \ssm B)$ is splittable (see \cref{def:splittable}).
\end{enumerate}
  These properties are proved by induction on the size of~$\nested$.  
  To initialize, observe that the rank~$0$ spine (with a single node~$\vertexSet$) is sent to the empty nested set for which Property~(b) above holds by \cref{def:block}.

  Assume now that these properties hold for a given nested set~$\nested$ and consider a bigger nested set~$\nested' \eqdef \nested \cup \{B\}$.
  Let~$\spine$ be the spine such that~$\spineToNested(\spine) = \nested$ and let~$X$ be a node of~$\spine$ such that the partition~$X = (X \cap B) \sqcup (X \ssm B)$ is splittable.
  Let~$\spine'$ denote the spine obtained by the splitting of~$(X \cap B) \sqcup (X \ssm B)$ in~$\spine$.
  We claim that $\spine$ satisfies Properties (a) and (b) above.
  
  Consider first an incoming arc~$\alpha$ of~$X$ in~$\spine$.
  Since~$B$ is compatible with~$\so[\alpha]$, and both~$X \cap B$ and~$X \ssm B$ are non-empty, we obtain that~$\so[\alpha] \subseteq B$ or~$\so[\alpha] \negDisjoint B$.
  Moreover, by \cref{def:nodeSplitting}, the arc~$\alpha$ is connected to the node~$X \cap B$ of~$\spine'$ if~$\so[\alpha] \subseteq B$, and to the node~$X \ssm B$ of~$\spine'$ if~$\so[\alpha] \negDisjoint B$.
  Similarly, for any outgoing arc~$\beta$ of~$X$ in~$\spine$, we have~$\so[\beta] \supseteq B$ or~$\so[\alpha] \posDisjoint B$, and the arc~$\beta$ is connected to the node~$X \ssm B$ of~$\spine'$ if~$\so[\beta] \supseteq B$, and to the node~$X \cap B$ of~$\spine'$ if~$\so[\beta] \posDisjoint B$.
  In particular, we have~$\so[\alpha] \subseteq B$ for each incoming arc~$\alpha$ of~$X \cap B$ in~$\spine'$, and~$\ta[\beta] \subseteq B$ for each outgoing arc~$\beta$ of~$X \cap B$ in~$\spine'$ (since~$\so[\beta] \posDisjoint B$ implies $\so[\beta] \cup B = \vertexSet$ hence $\ta[\beta] = \vertexSet \ssm \so[\beta] \subseteq B$).
  We thus obtain that~$\so[\gamma] \subseteq B$, and similarly that~$\ta[\gamma] \subseteq \vertexSet \ssm B$, from which we conclude that $\so[\gamma] = B$.
  Since~$\spine'$ is obtained by splitting a node of~$\spine$ into an arc~$\gamma$, we have~$\spineToNested(\spine') = \spineToNested(\spine) \cup \{\so[\gamma]\} = N \cup \{B\} = N '$.

  Consider now another block~$B'$ of~$\b{G}$ compatible with all blocks of~$N'$.
  Since~$B'$ is compatible with~$N \subseteq N'$, there is a node~$X'$ of~$\spine$ such that the partition~$X' = (X' \cap B') \sqcup (X' \ssm B')$ is splittable.
  If~$X \ne X'$, then~$X'$ is still a node of~$\spine'$ so that there is nothing to prove.
  If~$X = X'$, then the node~$X \cap B$ of~$\spine'$ suits if~$B \supseteq B'$ or~$B \negDisjoint B'$, while the node~$X \ssm B$ of~$\spine'$  suits if~$B \subseteq B'$ or~$B \posDisjoint B'$.
  \vincent{Check that, maybe add a sentence...}

  \para{Order-preserving}
  Finally, $\spineToNested$ is clearly order preserving as contracting one arc of a spine~$\spine$ removes the corresponding block of the nested set~$\spineToNested(\spine)$.
\end{proof}

\begin{remark}
  \label{rem:nestedToSpine}
  From the proof of \cref{prop:spineToNested}, we can give a direct description of the inverse map~$\nestedToSpine$ of the map~$\spineToNested$.
  Namely, consider a nested set~$N$ and two blocks~$B, B'$ of~$N$, and let~$\gamma, \gamma'$ denote the arcs of~$\spine \eqdef \nestedToSpine(N)$ such that~$\so[\gamma] = B$ and~$\so[\gamma'] = B'$.
  Then
  \begin{enumerate}[(i)]
    \item the target of~$\gamma$ and the source of~$\gamma'$ coincide iff~$B \subseteq B'$ and~$\nexists \, B'' \in \nested$ with~$B \subseteq B'' \subseteq B'$ or~$B \negDisjoint B'' \posDisjoint B'$,
    \item the targets of~$\gamma$ and~$\gamma'$ coincide iff~$B \negDisjoint B'$ and~$\nexists \, B'' \in \nested$ with~$B \subseteq B'' \negDisjoint B'$ or~${B \negDisjoint B'' \supseteq B'}$,
\item the sources of~$\gamma$ and~$\gamma'$ coincide iff~$B \posDisjoint B'$ and~$\nexists \, B'' \in \nested$ with~$B \posDisjoint B'' \subseteq B'$ or~${B \supseteq B'' \posDisjoint B'}$.
\end{enumerate}
This gives a description of the nodes of the spine~$\nestedToSpine(\nested)$ in terms of equivalence classes of blocks of~$\nested$ (by the relations above). It also gives a direct definition of the directed graph underlying~$\nestedToSpine(\nested)$ as a quotient of a collection of disjoint arcs labeled by~$\nested$ by identification of some of their endpoints. Finally, each node of~$\nestedToSpine(\nested)$ with incoming arcs~$A$ and outgoing arcs~$B$ is given by
\[
\bigg( \bigcap_{\beta \in B} \so[\beta] \bigg) \ssm \bigg( \bigcup_{\alpha \in A} \so[\alpha] \bigg) = \vertexSet \ssm \bigg( \bigcup_{\alpha \in A} \so[\alpha] \cup \bigcup_{\beta \in B} \ta[\beta] \bigg).
\]
\end{remark}

\begin{corollary}
  The nested complex~$\nestedComplex$ is a pure and flag simplicial complex.
\end{corollary}

%%%%%%%%%%%%%%%%%%%%%%%%%%%%%%%%%%%%%%%

\subsection{Flips}

\begin{definition}
  \label{def:flip}
  Let $\b{G}$ be a decorated block graph and let $\spine$ be a maximal spine on~$\b{G}$.
  Consider two vertices $X = \{x\}$ and $Y = \{y\}$ of $\spine$ related by an arc $\gamma$. 
  Let $\alpha$ be the incoming arc of $X$ such that~$\so[\alpha]$ and~$Y$ are contained in the same connected component of $G \ssm \Down[X]$, and let $\beta$ be the outgoing arc of $Y$ such that~$\ta[\beta]$ and~$X$ are contained in the same connected component of $G \ssm \Up[Y]$.
  We define $\spine'$ to be the spine obtained from $\spine$ by reversing the orientation of~$\gamma$, grafting the arc~$\alpha$ to~$Y$ and the arc~$\beta$ to~$X$.
  We say that $\spine'$ is obtained from $\spine$ by \defn{flipping} the arc~$\gamma$. 
\end{definition}

\vincent{Add picture of two corank $0$ spines connected by a flip.}

\begin{figure}[h!]
\begin{tikzpicture}[scale=1.6]
    
\node (N0) [circle,draw=none,minimum size=4mm,inner sep=0.1mm] at (-0.5,-0.5) {\small $x$};
\node (N1) [circle,draw=none,minimum size=4mm,inner sep=0.1mm] at (0.5,0.5) {\small $y$};
    
\node (o1) [circle,draw=none,minimum size=4mm,inner sep=0.1mm] at (-1.2,0.6) {\small $\beta_1$};
\node (od) [circle,draw=none,minimum size=4mm,inner sep=0.1mm] at (-0.8,0.6) {\small $\dots$};
\node (oj) [circle,draw=none,minimum size=4mm,inner sep=0.1mm] at (-0.3,0.6) {\small $\beta_k$};
    
\node (oj1) [circle,draw=none,minimum size=4mm,inner sep=0.1mm] at (0,1.62) {\small $\beta_{k+1}$};
\node (ojd) [circle,draw=none,minimum size=4mm,inner sep=0.1mm] at (0.5,1.6) {\small $\dots$};
\node (ol) [circle,draw=none,minimum size=4mm,inner sep=0.1mm] at (1,1.6) {\small $\beta_\ell$};

\node (2) [circle,draw=none,minimum size=4mm,inner sep=0.1mm] at (1.5,1.6) {\small $\boldsymbol{\beta}$};

\node (1) [circle,draw=none,minimum size=4mm,inner sep=0.1mm] at (-1.5,-1.6) {\small $\boldsymbol{\alpha}$};
    
\node (i1) [circle,draw=none,minimum size=4mm,inner sep=0.1mm] at (-1,-1.6) {\small $\alpha_1$};
\node (id) [circle,draw=none,minimum size=4mm,inner sep=0.1mm] at (-0.5,-1.6) {\small $\dots$};
\node (im) [circle,draw=none,minimum size=4mm,inner sep=0.1mm] at (0,-1.6) {\small $\alpha_i$};
    
\node (im1) [circle,draw=none,minimum size=4mm,inner sep=0.1mm] at (0.3,-0.62) {\small $\alpha_{i+1}$};
\node (imd) [circle,draw=none,minimum size=4mm,inner sep=0.1mm] at (0.8,-0.6) {\small $\dots$};
\node (ik) [circle,draw=none,minimum size=4mm,inner sep=0.1mm] at (1.2,-0.6) {\small $\alpha_j$};

\draw[->,thick] (1)--(N0); 
\draw[->,thick] (N1)--(2); 
    
\draw[->] (N0)--(o1); 
\draw[->] (N0)--(od);
\draw[->] (N0)--(oj);
      
\draw[->] (N0)--(N1) node[midway,left] {$\gamma$} ; 
\draw[->] (N1)--(oj1); 
\draw[->] (N1)--(ojd);
\draw[->] (N1)--(ol);
    
\draw[->] (i1)--(N0);
\draw[->] (id)--(N0);
\draw[->] (im)--(N0); 
    
\draw[->] (im1)--(N1); 
\draw[->] (imd)--(N1);
\draw[->] (ik)--(N1);
    
\end{tikzpicture}
\resizebox{0.1\linewidth}{!}{
\raisebox{5.5em}{
\begin{tikzpicture}
    \draw[->] (0,0)--(1,0); 
\end{tikzpicture}}}
\quad
\begin{tikzpicture}[scale=1.6]
    
\node (N0) [circle,draw=none,minimum size=4mm,inner sep=0.1mm] at (-0.5,0.5) {\small $x$};
\node (N1) [circle,draw=none,minimum size=4mm,inner sep=0.1mm] at (0.5,-0.5) {\small $y$};
      
\node (o1) [circle,draw=none,minimum size=4mm,inner sep=0.1mm] at (-1,1.6) {\small $\beta_1$};
\node (od) [circle,draw=none,minimum size=4mm,inner sep=0.1mm] at (-0.5,1.6) {\small $\dots$};
\node (oj) [circle,draw=none,minimum size=4mm,inner sep=0.1mm] at (0,1.6) {\small $\beta_k$};
      
\node (oj1) [circle,draw=none,minimum size=4mm,inner sep=0.1mm] at (0.3,0.62) {\small $\beta_{k+1}$};
\node (ojd) [circle,draw=none,minimum size=4mm,inner sep=0.1mm] at (0.8,0.6) {\small $\dots$};
\node (ol) [circle,draw=none,minimum size=4mm,inner sep=0.1mm] at (1.2,0.6) {\small $\beta_\ell$};

      
\node (i1) [circle,draw=none,minimum size=4mm,inner sep=0.1mm] at (-1.2,-0.6) {\small $\alpha_1$};
\node (id) [circle,draw=none,minimum size=4mm,inner sep=0.1mm] at (-0.8,-0.6) {\small $\dots$};
\node (im) [circle,draw=none,minimum size=4mm,inner sep=0.1mm] at (-0.3,-0.6) {\small $\alpha_i$};
      
\node (im1) [circle,draw=none,minimum size=4mm,inner sep=0.1mm] at (0,-1.62) {\small $\alpha_{i+1}$};
\node (imd) [circle,draw=none,minimum size=4mm,inner sep=0.1mm] at (0.5,-1.6) {\small $\dots$};
\node (ik) [circle,draw=none,minimum size=4mm,inner sep=0.1mm] at (1,-1.6) {\small $\alpha_j$};

\node (2) [circle,draw=none,minimum size=4mm,inner sep=0.1mm] at (-1.5,1.6) {\small $\boldsymbol{\beta}$};

\node (1) [circle,draw=none,minimum size=4mm,inner sep=0.1mm] at (1.5,-1.6) {\small $\boldsymbol{\alpha}$};

\draw[->,thick] (1)--(N1); 
\draw[->,thick] (N0)--(2); 
      
\draw[->] (N0)--(o1); 
\draw[->] (N0)--(od);
\draw[->] (N0)--(oj);
        
\draw[->] (N1)--(N0) node[midway,right] {$\gamma$} ; 
\draw[->] (N1)--(oj1); 
\draw[->] (N1)--(ojd);
\draw[->] (N1)--(ol);
      
\draw[->] (i1)--(N0);
\draw[->] (id)--(N0);
\draw[->] (im)--(N0); 
      
\draw[->] (im1)--(N1); 
\draw[->] (imd)--(N1);
\draw[->] (ik)--(N1);
\end{tikzpicture}
\caption{A spine flip.}
\end{figure}

The fact that $\spine'$ is indeed a spine is immediate from the definitions. Contracting $\gamma$ in either $\spine$ or $\spine'$, we obtain the same spine $\spine''$. 

\begin{lemma} 
  \label{lemma:coveringpair} 
  The spines~$\spine$ and $\spine'$ are the only spines covering $\spine''$ in the spine poset~$\spinePoset$.
\end{lemma}

\begin{proof}
  This follows from \cref{prop:nodeSplitting,prop:contractionSplitting}.
  \vincent{I don't understand why it is so simple compared to the proof in my paper...}
\end{proof}
  
\begin{definition}
  \label{def:flipGraph}
  The \defn{flip graph} is the graph whose vertices are the maximal spines on~$\b{G}$ and whose egdes are the flips between them.
  In other words, it is the facet-ridge graph of the nested complex~$\nestedComplex$.
\end{definition}

\begin{corollary} 
   The nested complex~$\nestedComplex$ is a closed pseudo-manifold. 
\end{corollary}
  
\begin{definition}
  \label{def:increasingFlip}
  Fix a total order~$\prec$ on~$\vertexSet$.
  The flip from~$\spine$ to~$\spine'$ in \cref{def:flip} is \defn{$\prec$-incresing} if~$x \prec y$ and \defn{$\prec$-decreasing} if $x \succ y$.
  The \defn{$\prec$-increasing flip graph} is the directed graph whose vertices are the maximal spines on~$\b{G}$ and whose arcs are the increasing flips between them.
  The \defn{$\prec$-increasing flip poset} is the transitive closure of the $\prec$-increasing flip graph.
\end{definition}

\begin{example}
  \label{exm:flipPosets}
  For instance, the increasing flip poset is (isomorphic to):
  \begin{enumerate}[(i)]
    \item the \defn{weak order} on permutations of~$\vertexSet$ when $\b{G}$ is complete or undecorated.
    \item the \defn{flip poset on maximal tubings} of~$G$ studied in~\cite{BarnardMcConville} when~$\b{G}$ is down decorated (in particular the classical \defn{Tamari lattice}~\cite{Tamari, HuangTamari} when~$\b{G}$ is a down decorated path).
    \item the \defn{acyclic reorientation lattice} of~$G$~\cite{Pilaud-acyclicReorientationLattices} when~$\b{G}$ is fully decorated (in particular the \defn{boolean lattice} when~$\b{G}$ is a fully decorated tree).
    \item the various \defn{permutree lattices} of~\cite{PilaudPons-permutrees} when~$G$ is a path (in particular the various type~$A$ \defn{Cambrian lattices}~\cite{Reading-CambrianLattices} for up-down decorated paths).
  \end{enumerate}
\end{example}

\begin{remark}
  The question of whether or not the increasing flip poset is a lattice is a difficult question. It depends on the decoration $\decoration$. We know that if $G$ is a path, it is always the case \cite{PilaudPons-permutrees}, for $G$ a block graph it is generally not true. See \cite{BarnardMcConville}. \guillaume{Nos calculs?; donner un exemple, un contre-exemple}\vincent{improve this remark.}
\end{remark}

\vincent{We need a picture of increasing flip poset here. Take the natural order for $\prec$.}

%%%%%%%%%%%%%%%%%%%%%%%%%%%%%%%%%%%%%%%

\subsection{Blossoming spines}
\label{subsec:blossomingSpines}

In the next section, we aim at describing natural surjections between spines of certain decorated block graphs.
For this, it is convenient to add further information to a spine on~$\b{G}$.
Here, we need to manipulate both the vertices and the cliques of~$G$, thus we consider a maple tree~$M$ corresponding to the block graph~$G$.

\begin{definition}
  \label{def:labeledSpine}
  Let~$\b{M} \eqdef (M, \decoration)$ be a decorated maple tree and~$\b{G} \eqdef (G, \decoration)$ be the decorated block graph obtained by tapping~$M$.
  In a spine~$\spine$ on~$\b{G}$, we label
  \begin{itemize}
    \item each node~$X$ by the connected component~$\labeling(X)$ of~$M \ssm \big( \Up[{\so[X]}] \cup \Down[{\ta[X]}] \big)$ containing~$X$,
    \item each arc~$\gamma$ by the intersection~$\labeling(\gamma)$ of the connected component of~$M \ssm \Up[{\so[\gamma]}]$ containing~$\ta[\gamma]$ with the connected component of~$M \ssm \Down[{\ta[\gamma]}]$ containing~$\so[\gamma]$.
  \end{itemize}
%  We denote by~$\labeling$ the label map on nodes and arcs of~$\spine$.
\end{definition}

We now add a little more to a labeled spine.

\begin{definition}
  \label{def:blossomingTree}
  A \defn{directed blossoming tree} is a directed tree~$\spine$ with some additional 
  \begin{itemize}
    \item \defn{incoming blossoms} (arcs joining an empty node to a normal node) and 
    \item \defn{outgoing blossoms} (arcs joining a normal node to an empty node).
  \end{itemize}
  A \defn{cut} of a directed blossoming tree~$\spine$ is a set~$\Gamma$ of nodes, arcs and blossoms of~$\spine$ such that any directed path joining the source of an incoming blossom of~$\spine$ to the target of an outgoing blossom of~$\spine$ contains precisely one element of~$\Gamma$.
%  When the nodes of~$\spine$ are sets, the \defn{source set}~$\so[\Gamma]$ (resp.~\defn{target set}~$\ta[\Gamma]$) of~$\Gamma$ is the union of the nodes of all connected components of~$\spine \ssm \Gamma$ which contain the source (resp.~target) of at least one arc of~$\Gamma$.
  When the nodes of~$\spine$ are sets, the \defn{source set}~$\so[\Gamma]$ (resp.~\defn{target set}~$\ta[\Gamma]$) of~$\Gamma$ is the union of the nodes of all connected components of~$\spine \ssm \Gamma$ which contain the source of an incoming blossom.
  For instance, the set of incoming (resp.~outgoing) blossoms is a cut of~$\spine$ whose source (resp.~target) set is empty and whose target (resp.~source) set is the union of the nodes of~$\spine$.
\end{definition}

%  A \defn{sweep sequence} of~$\spine$ is a sequence~$\Gamma_0, \dots, \Gamma_p$ of cuts of~$\spine$ such that $\Gamma_0$ is the cut formed by all incoming blossoms of~$\spine$ while $\Gamma_p$ is the cut formed by all outgoing blossoms of~$\spine$, and for all~$i \in [p]$, there is a node~$X_i$ of~$\spine$ such that~$\Gamma_{i-1} \ssm \so[X_i] = \Gamma_i \ssm \ta[X_i]$.

\begin{definition}
  \label{def:blossomingSpine}
  Let~$\b{M} \eqdef (M, \decoration)$ be a decorated maple tree, let~$\b{G} \eqdef (G, \decoration)$ be the decorated block graph obtained by tapping~$M$, and let~$\spine$ be a spine on~$\b{G}$.
  The \defn{blossoming spine}~$\spine\blossom$ is the labeled blossoming tree obtained from the labeled spine~$\spine$ by a adding at each node~$X$ of~$\spine$ an incoming (resp.~outgoing) blossom labeled by~$C$ for each connected component~$C$ of~$M \ssm \Down[{X \cup \ta[X]}]$ (resp.~of~$M \ssm \Up[{X \cup \so[X]}]$) adjacent to~$X$ such that there is no incoming arc~$\alpha$ (resp.~outgoing arc~$\beta$) of~$X$ with~$\so[\alpha]$ (resp.~$\ta[\beta]$) contained in~$C$.
  \vincent{give two examples of blossoming spines on the examples of spines given before.}
\end{definition}

%\begin{definition}
%  \label{def:blossomingSpine}
%  Let~$\b{M} \eqdef (M, \decoration)$ be a decorated maple tree and~$\b{G} \eqdef (G, \decoration)$ be the decorated block graph obtained by tapping~$M$.
%  The \defn{blossoming}~$\spine\blossom$ of a spine~$\spine$ on~$\b{G}$ is a directed blossoming tree with the same nodes and arcs as~$\spine$ and some additional blossoms, and where all nodes and arcs (including blossoms) are additionally labeled by a maple subtrees of~$M$. \vincent{Maybe define maple subtree.}
%  It is obtained from~$\spine$ by
%  \begin{enumerate}[(i)]
%    \item labeling each node~$X$ of~$\spine$ by the connected component of~$M \ssm (\Up[{\so[X]}] \cup \Down[{\ta[X]}])$ containing~$X$,
%    \item labeling each arc~$\gamma$ of~$\spine$ by the intersection of the connected component of~$M \ssm \Up[{\so[\gamma]}]$ containing~$\ta[\gamma]$ with the connected component of~$M \ssm \Down[{\ta[\gamma]}]$ containing~$\so[\gamma]$,
%    \item adding at each node~$X$ of~$\spine$ an incoming (resp.~outgoing) blossom labeled by~$C$ for each connected component~$C$ of~$M \ssm \Down[{X \cup \ta[X]}]$ (resp.~of~$M \ssm \Up[{X \cup \so[X]}]$) adjacent to~$X$ such that there is no incoming arc~$\alpha$ (resp.~outgoing arc~$\beta$) of~$X$ with~$\so[\alpha]$ (resp.~$\ta[\beta]$) contained in~$C$.
%  \end{enumerate}
%  We call \defn{blossoming spines} of~$M$ the blossomings of the spines of~$G$.
%  \vincent{give two examples of blossoming spines on the examples of spines given before.}
%\end{definition}

By definition, the labels of the nodes and the arcs of~$\spine\blossom$ are some maple subtrees of~$M$.
Note that a maple subtree of~$M$ is completely determined by its leaves, hence by its red vertices.
We thus simplify all pictures of blossoming spines by writing only the red (letters) vertices of each label.
\vincent{We could even just write the leaves...}

%\begin{lemma}
%  \label{lem:connectedComponents}
%  Let~$\b{M} \eqdef (M, \decoration)$ be a decorated maple tree, let~$\b{G} \eqdef (G, \decoration)$ be the decorated block graph obtained by tapping~$M$, and let~$\spine$ be a spine on~$\b{G}$.
%  Then the labels of the incoming (resp.~outgoing) blossoms of~$S$ are precisely the intersections of~$R$ with the connected components of~${M \ssm \Down[B]}$ (resp.~of~${M \ssm \Up[B]}$).
%\end{lemma}
%
%\begin{proof}
%  \vincent{todo}
%\end{proof}

\begin{proposition}
  \label{prop:blossomingSpine}
  Let~$\b{M} \eqdef (M, \decoration)$ be a decorated maple tree, let~$\b{G} \eqdef (G, \decoration)$ be the decorated block graph obtained by tapping~$M$, and let~$\spine$ be a spine on~$\b{G}$.
  For any cut~$\Gamma$ of~$\spine\blossom$, the labels of the nodes, arcs and blossoms of~$\Gamma$ are precisely the connected components of~$M \ssm \big( \Up[{\so[\Gamma]}] \cup \Down[{\ta[\Gamma]}] \big)$.
\end{proposition}

\begin{proof}
  \vincent{Not clear how to argue that properly...}
%  We first prove the result for the cut~$\Gamma$ of~$\spine$ given by all incoming blossoms of~$\spine$.
%  \vincent{todo}
%  Let~$\Gamma$ and~$\Gamma'$ be two cuts of~$\spine$ which differ by sweeping a node~$X$ of~$\spine$, meaning that $\Gamma \ssm \alpha(X) = \Gamma' \ssm \beta(X)$ where~$\alpha(X)$ and~$\beta(X)$ respectively denote the incoming and outgoing arcs of~$X$.
%  \vincent{todo}
\end{proof}

\begin{figure}[h!]
\begin{tikzpicture}
  \node (0) at (0, 0) {$\updown[2] \up[5] 6$};
  \node (1) at (-3, 1) {$ade$};
  \node (2) at (-1, 1) {$hi$};
  \node (3) at (1, 1) {$fgjk$};
  \node (4) at (3, 1) {$bc$};
  \node (5) at (3, 3) {$bc$};
  \node (6) at (1.5, 3) {$k$};
  \node (7) at (0.5, 3) {$fgj$};
  \node (8) at (-1, 3) {$hi$};
  \node (9) at (-3, 3) {$ade$};
  \node (10) at (1, 2) {$\up[9]$};
  \node (11) at (-3, 2) {$14$};
  \node (12) at (-1, -1) {$adefghijk$};
  \node (13) at (1, -1) {$bc$};
  \node (14) at (-1, -2) {$\down[7] 8$};
  \node (15) at (1, -2) {$3$};
  \node (16) at (-2, -3) {$adefghjk$};
  \node (17) at (0, -3) {$i$};
  \node (18) at (1, -3) {$bc$};
  \draw[->] (18)--(15);
  \draw[->] (17)--(14);
  \draw[->] (16)--(14);
  \draw[->] (15)--(13);
  \draw[->] (14)--(12);
  \draw[->] (13)--(0);
  \draw[->] (12)--(0);
  \draw[->] (0)--(1);
  \draw[->] (0)--(2);
  \draw[->] (0)--(3);
  \draw[->] (0)--(4);
  \draw[->] (1)--(11);
  \draw[->] (2)--(8);
  \draw[->] (3)--(10);
  \draw[->] (4)--(5);
  \draw[->] (11)--(9);
  \draw[->] (10)--(7);
  \draw[->] (10)--(6);
  \draw[dotted] (-5,-3)--(16)--(17)--(18)--(5,-3);
  \draw[dashed] (-5,-2)--(14)--(15)--(5,-2);
  \draw[dotted] (-5,-1)--(12)--(13)--(5,-1);
  \draw[dashed] (-5,0)--(0)--(5,0);
  \draw[dotted] (-5,1)--(1)--(2)--(3)--(4)--(5,1);
  \draw[dashed] (-5,2)--(11)--(10)--(5,2);
  \draw[dotted] (-5,3)--(9)--(8)--(7)--(6)--(5)--(5,3);
  \node (20) at (5.5, -3) {$\down[2] \down[7]$};
  \node (21) at (5.5, -2) {$\down[2]$};
  \node (22) at (5.5, -1) {$\down[2]$};
  \node (23) at (5.5, 0) {};
  \node (24) at (5.5, 1) {$\up[2] \up[5]$};
  \node (25) at (5.5, 2) {$\up[2] \up[5]$};
  \node (26) at (5.5, 3) {$\up[2] \up[5] \up[9]$};
\end{tikzpicture}
\caption{A blossoming spine on the maple tree of \cref{fig:mapleBlock}, reconstructed from $\pi = 3 \down[7] 8 | \updown[2] \up[5] 6 | 149$.}
\label{fig:blossomingSpine}
\end{figure}

%%%%%%%%%%%%%%%%%%%%%%%%%%%%%%%%%%%%%%%

\subsection{Refinement and surjection maps}
\label{subsec:surjectionMaps}

\begin{definition}
  \label{def:refinement}
  For two decorations~$\decoration$ and~$\decoration'$ on a set~$\vertexSet$, we say that~$\decoration$ \defn{refines}~$\decoration'$ (and that $\decoration'$ \defn{coarsens} $\decoration$) if~$\Down[\vertexSet] \subseteq \Down[\vertexSet][\decoration']$ and $\Up[\vertexSet] \subseteq \Up[\vertexSet][\decoration']$.
  For two decorated block graphs~$\b{G} \eqdef (G, \decoration)$ and~$\b{G}' \eqdef (G', \decoration')$, we say that~$\b{G}$ \defn{refines}~$\b{G}'$ (and that $\b{G}'$ \defn{coarsens} $\b{G}$) if~$G = G'$ and $\decoration$ refines~$\decoration'$.
\end{definition}

We now define a natural surjection~$\surjectionSpines$ from the spines on~$\b{G}$ to the spines on~$\b{G}'$ when~$\b{G}$ refines~$\b{G}'$.
%Here, we need the labeling~$\labeling$ of the nodes and arcs of the spines described in \cref{def:labeledSpine}, hence we work with the maple tree~$M$ corresponding to the block graph~$G$.
For this, it is convenient to add further information to a spine on~$\b{G}$.
Here, we need to manipulate both the vertices and the cliques of~$G$, thus we consider a maple tree~$M$ corresponding to the block graph~$G$.

\begin{definition}
  \label{def:labeledSpine}
  Let~$\b{M} \eqdef (M, \decoration)$ be a decorated maple tree and~$\b{G} \eqdef (G, \decoration)$ be the decorated block graph obtained by tapping~$M$.
  In a spine~$\spine$ on~$\b{G}$, we label
  \begin{itemize}
    \item each node~$X$ by the connected component~$\labeling(X)$ of~$M \ssm \big( \Up[{\so[X]}] \cup \Down[{\ta[X]}] \big)$ containing~$X$,
    \item each arc~$\gamma$ by the intersection~$\labeling(\gamma)$ of the connected component of~$M \ssm \Up[{\so[\gamma]}]$ containing~$\ta[\gamma]$ with the connected component of~$M \ssm \Down[{\ta[\gamma]}]$ containing~$\so[\gamma]$.
  \end{itemize}
%  We denote by~$\labeling$ the label map on nodes and arcs of~$\spine$.
\end{definition}

\begin{remark}
  It follows from \cref{coro:splittablePartitions} that $\labeling(X)$ and~$\labeling(\gamma)$ are well-defined.
  Moreover, they are both non-empty since~$\labeling(X)$ contains~$X$ and~$\labeling(\gamma)$ contains all red vertices of~$M$ which are neighbors to both~$\so[\gamma]$ and~$\ta[\gamma]$.
  We will see in \cref{subsec:blossomingSpines} that these labels behave properly with respect to cuts of~$\spine$.
  We postpone this property as it is not needed for the purposes of this section.
\end{remark}

Using the labeling of \cref{def:labeledSpine}, we are now ready to define the surjection~$\surjectionSpines$.

\begin{definition}
  \label{def:refinementSpines}
  Let~$M$ be a maple tree and~$G$ be the block graph obtained by tapping~$M$.
  Consider two decorations~$\decoration, \decoration'$ on~$B$ such that~$\decoration$ refines~$\decoration'$.
  To a spine~$\spine$ on~$\b{G} \eqdef (G, \decoration)$, we associate the spine~$\spine'$ on~$\b{G}' \eqdef (G, \decoration')$ obtained by contracting the empty nodes in the directed tree with
  \begin{itemize}
    \item a node~$n_X^C \eqdef X \cap C$ for each node~$X$ of~$\spine$ and each connected component~$C$ of \linebreak ${M \ssm \big( \Up[{\so[X]}][\decoration'] \cup \Down[{\ta[X]}][\decoration'] \big)}$ contained in~$\labeling(X)$,
    \item an arc~$\varepsilon_\gamma^C$ joining the unique node~$n_X^D$ with~$C \subseteq D$ to the unique node~$n_Y^{E}$~with~${C \subseteq E}$ for each arc~$\gamma$ of~$\spine$ joining~$X$ to~$Y$ and each connected component~$C$ of~$M \ssm \big( \Up[{\so[\gamma]}][\decoration'] \cup \Down[{\ta[\gamma]}][\decoration'] \big)$ contained in~$\labeling(\gamma)$.
  \end{itemize}
  We denote by~$\surjectionSpines$ the map that sends the spine $\spine$ on~$\b{G}$ to the spine~$\spine'$ on~$\b{G}'$.
\end{definition}

%\begin{definition}
%  \label{def:refinementSpines}
%  Let~$M$ be a maple tree with red vertices~$R$ and blue vertices~$B$, and let~$G$ be the block graph obtained by tapping~$M$.
%  Consider two decorations~$\decoration, \decoration'$ on~$B$ such that~$\decoration$ refines~$\decoration'$.
%  For a blossoming spine~$\spine$ on~$\b{G}$, we construct a spine~$\spine'$ on~$\b{G}'$ in three steps:
%  \begin{itemize}%[wide, labelwidth=!, labelindent=12pt]
%    \item \textbf{construct}
%    \begin{itemize}%[wide, labelwidth=!, labelindent=20pt]
%      \item a node~$n_X^C \eqdef X \cap C$ labeled by~$\labeling(n_X^C) \eqdef C$ for each node~$X$ of~$\spine$ and each component~$C$ of~$M \ssm \big( \Up[{\so[X]}] \cup \Down[{\ta[X]}] \big)$ contained in~$\labeling(X)$,
%      \item an arc~$\varepsilon_\gamma^D$ labeled by~$\labeling(\varepsilon_\gamma^D) \eqdef D$ for each arc~$\gamma$ of~$\spine$ and each component~$D$ of~$M \ssm \big( \Up[{\so[\gamma]}] \cup \Down[{\ta[\gamma]}] \big)$ contained in~$\labeling(\gamma)$,
%    \end{itemize}
%%    \item \textbf{attach} the arcs~$\varepsilon_\gamma^D$ with~$C \subseteq D$ as incoming (resp.~outgoing)~arcs of the node~$n_{\ta[\gamma]}^C$ (resp.~$n^{\so[\gamma]}_D$), for each arc~$\gamma$ of~$\spine$, 
%    \item \textbf{attach} the arc~$\varepsilon_\gamma^D$ as an incoming (resp.~outgoing)~arc of the node~$n_X^C$, when~$C \supseteq D$ and~$X = \ta[\gamma]$ (resp.~$X = \so[\gamma]$),
%%    \item \textbf{relabel} the arc~$\alpha^i_C$ with~$C \cap R$ and the node~$\alpha^i_D$ with~$D \cap \pi_i$,
%    \item \textbf{contract} into a single arc any path of arcs passing through empty nodes.
%%    	\item \textbf{erase} the blossoms and the labels on the arcs in the resulting blossoming spine to obtain a spine on~$\b{G}$.
%  \end{itemize}
%\end{definition}

\begin{remark}
  Note that: % \cref{def:refinementSpines} needs a few justifications:
  \begin{enumerate}[(i)]
    \item Since~$X$ is contained both in~$\labeling(X)$ and in ${M \ssm \big( \Up[{\so[X]}][\decoration'] \cup \Down[{\ta[X]}][\decoration'] \big)}$, the nodes~$n_X^C$ partition the node~$X$. Hence, since the nodes of~$\spine$ partition~$\vertexSet$, the nodes of~$\spine'$ partition~$\vertexSet$.
    \item For an arc~$\gamma$ of~$\spine$ joining~$X$ to~$Y$ and a connected component~$C$ of~$M \ssm \big( \Up[{\so[\gamma]}][\decoration'] \cup \Down[{\ta[\gamma]}][\decoration'] \big)$ there is a unique connected component~$D$ of ${M \ssm \big( \Up[{\so[X]}][\decoration'] \cup \Down[{\ta[X]}][\decoration'] \big)}$ containing~$C$, and a unique connected component~$E$ of ${M \ssm \big( \Up[{\so[Y]}][\decoration'] \cup \Down[{\ta[Y]}][\decoration'] \big)}$ containing~$C$. Indeed... todo.
    \item By construction, the map~$n_X^C \mapsto X$ is a graph homomorphism from the transitive closure of~$\spine'$ to the transitive closure of~$\spine$. Moreover, $\spine'$ is the contraction minimal spine on~$\b{G}'$ such that the inclusion map defines a graph homomorphism from the transitive closure of~$\spine'$ to the transitive closure of~$\spine$.
    \item The directed graph before contraction is a tree. Indeed... todo
    \item The directed tree after contraction of the empty nodes is a spine on~$\b{G}'$. Indeed... todo.
  \end{enumerate}
\end{remark}

\begin{proposition}
  \label{prop:refinementSpines}
  If~$\b{G}$ refines~$\b{G}'$, then the map~$\surjectionSpines$ is a surjection from the spines on~$\b{G}$ to the spines on~$\b{G}'$.
  Moreover, the spine~$\surjectionSpines(\spine')$ is the unique spine on~$\b{G}'$ of which the spine~$\spine$ on~$\b{G}$ is an extension.
  \vincent{this is incorrect at the moment. I want to say that it is the contraction minimal spine of~$\b{G}'$ of which~$\spine$ is an extension.}
  \vincent{add surjective}
\end{proposition}

\begin{proof}
  First, we need to prove that~$\spine'$ is indeed a spine on~$\b{G}'$.
  \vincent{todo}
\end{proof}

As the labels of a spine can be described directly from the spine, \cref{def:refinementSpines,prop:refinementSpines} define a surjection map from the spines of~$\b{G}$ to the spines of~$\b{G}'$ which we denote by~$\surjectionSpines$.

We now compare the nested complexes~$\nestedComplex[\b{G}]$ and~$\nestedComplex[\b{G}']$

\begin{proposition}
  \label{prop:refinementBlocks}
  If~$\b{G}$ refines~$\b{G}'$, then
  \begin{itemize}
    \item any block of~$\b{G}'$ is a block of~$\b{G}$,
    \item the nested complex~$\nestedComplex[\b{G}']$ contains the subcomplex of~$\nestedComplex[\b{G}]$ induced by the blocks of~$\b{G}'$.
  \end{itemize}
\end{proposition}

\begin{proof}
  \vincent{todo}
\end{proof}

%%%%%%%%%%%%%%%%%%%%%%%%%%%%%%%%%%%%%%%

\subsection{Surjection maps and rewriting rules}
\label{subsec:surjectionMaps}

\vincent{I want to do that directly from spines of~$\b{G}$ to spines of~$\b{G}'$ when~$\b{G}$ refines~$\b{G}'$.}

Recall that an \defn{ordered partition} of a set~$B$ is a (ordered) sequence~$\pi \eqdef \pi_1 | \dots | \pi_p$ of non-empty (unordered) subsets of~$B$ which form a partition of~$B$.

%\begin{definition}
%  \label{def:sweepingAlgorithm}
%  Let~$M$ be a maple tree with red vertices~$R$ and blue vertices~$B$.
%  The \defn{sweeping algorithm} is a procedure that maps an ordered partition~$\pi \eqdef \pi_1 | \dots | \pi_p$ of~$B$ into a blossoming spine~$\spine(\pi)$ on~$M$.
%  It follows a sweep sequence~$\Gamma_0, \dots \Gamma_p$ of~$\spine(\pi)$ and constructs~$\spine(\pi)$ by maintaining before step~$i$ the blossoming forest induced by the nodes of~$\spine(\pi)$ contained in~$\pi_1 \cup \dots \cup \pi_{i-1}$.
%  The iterative procedure is as follows:
%  \begin{itemize}
%    \item start with the cut~$\Gamma_0$ formed by an incoming blossom labeled~$R \cap C$ for each connected component~$C$ of~$M \ssm \Down[B]$,
%    \item at step~$i$:
%      \begin{itemize}
%        \item create a node~$X$ for each non-empty intersection of~$\pi_i$ with a connected component of~$M \ssm (\Up[\pi_1] \cup \dots \cup \Up[\pi_{i-1}] \cup \Down[\pi_{i+1}] \cup \dots \cup \Down[\pi_p])$,
%        \item attach as incoming arcs of~$X$ the outgoing blossoms of~$\Gamma_{i-1}$ whose labels are adjacent to~$X$ in~$M$,
%        \item create at~$X$ one outgoing blossom labeled by~$R \cap C$ for each connected component~$C$ of~$M \ssm (\Up[\pi_1] \cup \dots \cup \Up[\pi_i] \cup \Down[\pi_{i+1}] \cup \dots \cup \Down[\pi_p])$,
%        \item define the new cut~$\Gamma_i$.
%      \end{itemize}
%    \item end with the cut~$\Gamma_p$ with an outgoing blossom for each connected component of~$M \ssm \Up[B]$.
%  \end{itemize}
%  \vincent{improve that}
%\end{definition}

\begin{definition}
  \label{def:surjection}
  Let~$M$ be a maple tree with red vertices~$R$ and blue vertices~$B$.
  For an ordered partition~$\pi \eqdef \pi_1 | \dots | \pi_p$ of~$B$, we construct a blossoming spine~$\partitionToSpine(\pi)$ on~$M$ in four steps:
  \vincent{Do we want ordered partitions of surjections?}
  \begin{itemize}[wide, labelwidth=!, labelindent=5pt]
    \item \textbf{construct}
    \begin{itemize}[wide, labelwidth=!, labelindent=10pt]
      \item an arc~$\alpha^i_D$ for each $0 \le i \le p$ and each component~$D$ of~$M \ssm (\Up[\pi_1] \cup \dots \cup \Up[\pi_i] \cup \Down[\pi_{i+1}] \cup \dots \cup \Down[\pi_p])$,
      \item a node~$n^i_C$ for each $1 \le i \le p$ and each component~$C$ of~$M \ssm (\Up[\pi_1] \cup \dots \cup \Up[\pi_{i-1}] \cup \Down[\pi_{i+1}] \cup \dots \cup \Down[\pi_p])$,
    \end{itemize}
    \item \textbf{attach} the arcs~$\alpha^{i-1}_C$ (resp.~$\alpha^i_C$) with~$C \subseteq D$ as incoming (resp.~outgoing)~arcs of the node~$n^i_D$,
    \item \textbf{relabel} the arc~$\alpha^i_C$ with~$C \cap R$ and the node~$\alpha^i_D$ with~$D \cap \pi_i$,
    \item \textbf{contract} into a single arc any path of arcs passing through nodes with empty~labels.
  \end{itemize}
\end{definition}

\begin{proposition}
  \label{prop:surjection}
  For any ordered partition~$\pi \eqdef \pi_1 | \dots | \pi_p$, the directed graph~$\partitionToSpine(\pi)$ constructed in \cref{def:surjection} is a blossoming spine on~$M$.
  Moreover, $\pi$ is an extension of~$\partitionToSpine(\pi)$.
  \vincent{explain}
\end{proposition}

\begin{remark}
  Note that if an ordered partition~$\pi$ of~$\vertexSet$ has~$k$ parts, then the spine~$\partitionToSpine(\pi)$ on~$\b{G}$ has at least $k$ nodes (it can have more, see \cref{fig:blossomingSpine}). In particular, the permutations of~$\vertexSet$ are sent to the corank~$0$ spines on~$\b{G}$.
\end{remark}

\begin{remark}
  The construction of \cref{def:surjection} can also be seen as a sweeping algorithm.
  \vincent{todo}
\end{remark}

\vincent{Specialize to permutations?}
\vincent{Rewriting rules...}

We now consider the equivalence relation~$\equiv_\b{G}$ whose classes are the fibers of the map~$\partitionToSpine$.

\begin{definition}
  \label{def:equivalenceRelation}
  Let~$\equiv_\b{G}$ denote the equivalence relation on ordered partitions of~$\vertexSet$ defined by~$\pi \equiv_\b{G} \pi'$ if and only if $\partitionToSpine(\pi) = \partitionToSpine(\pi')$.
\end{definition}

\begin{example}
  \label{exm:equivalenceRelations}
  For instance, the equivalence relation~$\equiv_\b{G}$ on permutations is
  \begin{enumerate}[(i)]
    \item the \defn{trivial relation} when $\b{G}$ is complete or undecorated.
    \item the various \defn{permutree relations} of~\cite{PilaudPons-permutrees} when~$G$ is a path (in particular the \defn{sylvester relation} for a down decorated path, and the \defn{Cambrian relations} for up-down decorated~paths).
  \end{enumerate}
  Note that in contrast to these specific examples, the equivalence relation~$\equiv$ has no reason to be a lattice congruence of the weak order.
\end{example}

\begin{proposition}
  \label{prop:equivalenceRelation}
  The equivalence relation~$\equiv_\b{G}$ on ordered partitions of~$\vertexSet$ is the transitive closure of the elementary relations
  \[
    \pi_1 | \dots | \pi_k | \pi_{k+1} | \dots | \pi_p \equiv_\b{G} \pi_1 | \dots | \pi_k \cup \pi_{k+1} | \dots | \pi_p
  \]
  where~$\pi_k$ and~$\pi_{k+1}$ belong to distinct connected components of~$G \ssm (\Up[\pi_1] \cup \dots \cup \Up[\pi_{k-1}] \cup \Down[\pi_{k+2}] \cup \dots \cup \Down[\pi_p])$.
\end{proposition}

\begin{proof}
  Follows from \cref{def:surjection}.
\end{proof}

\begin{remark}
  Specializing to permutations, the equivalence relation~$\equiv_\b{G}$ on permutations of~$\vertexSet$ is the transitive closure of the elementary relations
  \[
    \sigma_1 \dots \sigma_k \sigma_{k+1} \dots \sigma_{|\vertexSet|} \equiv_\b{G} \sigma_1 \dots \sigma_{k+1} \sigma_k \dots \sigma_{|\vertexSet|}
  \]
  where~$\sigma_k$ and~$\sigma_{k+1}$ belong to distinct connected components of~$G \ssm (\Up[{\sigma^{-1}([k-1])}] \cup \Down[{\sigma^{-1}([|\vertexSet|] \ssm [k+1])}])$.
  \vincent{Here, I would prefer to say that there is a vertex of~$G$ in between $\sigma_k$ and~$\sigma_{k+1}$...}
\end{remark}


%%%%%%%%%%%%%%%%%%%%%%%%%%%%%%%%%%%%%%%

\subsection{Refinement}
\label{subsec:refinement}

\begin{definition}
  \label{def:refinement}
  %Let~$\b{G}$ and~$\b{G}'$ be two decorated block graphs.
  We say that~$\b{G}$ \defn{refines}~$\b{G}'$ (and that $\b{G}'$ \defn{coarsens} $\b{G}$) if they are decorations of the same underlying block graph and $\Down[\vertexSet] \subseteq \Down[\vertexSet']$ while $\Up[\vertexSet] \subseteq \Up[\vertexSet']$.
\end{definition}

\begin{proposition}
  \label{prop:refinementBlocks}
  If~$\b{G}$ refines~$\b{G}'$, then
  \begin{itemize}
    \item any block of~$\b{G}'$ is a block of~$\b{G}$,
    \item the nested complex~$\nestedComplex[\b{G}']$ contains the subcomplex of~$\nestedComplex[\b{G}]$ induced by the blocks of~$\b{G}'$.
  \end{itemize}
\end{proposition}

\begin{proof}
  \vincent{todo}
\end{proof}

\begin{proposition}
  \label{prop:refinementEquivalenceRelations}
  If~$\b{G}$ refines~$\b{G}'$, then~$\equiv_\b{G}$ refines~$\equiv_{\b{G}'}$.
\end{proposition}

\begin{proof}
  Follows from \cref{prop:equivalenceRelation}.
\end{proof}

For any ordered partitions~$\pi$ of~$\vertexSet$ and~$\pi'$ of~$\vertexSet'$ corresponding to the same ordered partition of the underlying block graph of~$\b{G}$ and~$\b{G}'$, \cref{prop:refinementEquivalenceRelations} affirms that the spine~$\spine = \partitionToSpine(\pi')$ of~$\b{G}'$ only depends on the spine~$\spine' = \partitionToSpine(\pi)$ of~$\b{G}$.
This yields a natural surjection~$\surjectionSpines$ from spines on~$\b{G}$ to spines on~$\b{G}'$, which is clearly a poset morphism from~$\spinePoset[\b{G}]$ to~$\spinePoset[\b{G}']$.
\vincent{Is it possible to define directly this surjection. This would save the surjection from ordered partitions to spines...}

%\cref{prop:refinementEquivalenceRelations} yields a natural surjection~$\surjectionSpines$ from spines on~$\b{G}$ to spines on~$\b{G}'$.
%Namely, for a spine~$\spine$ on~$\b{G}$, pick any ordered partition~$\pi$ of~$\vertexSet$ such that~$\spine = \partitionToSpine(\pi)$, let~$\pi'$ be the corresponding ordered partition of~$\vertexSet'$ where only the decoration differ, and let~$\spine' = \partitionToSpine(\pi')$.
%This surjection is clearly a poset morphism from~$\spinePoset[\b{G}]$ to~$\spinePoset[\b{G}']$.

\begin{example}
  \vincent{Surjections from spines to acyclic orientations of~$G$.}
\end{example}

%%%%%%%%%%%%%%%%%%%%%%%%%%%%%%%%%%%%%%%

\subsection{Links}
\label{subsec:links}

\vincent{need to explain links in terms of spines as well...}

\begin{proposition}
  \label{prop:links}
  The link of any face of the nested complex~$\nestedComplex$ of a decorated block graph~$\b{G}$ is a join of nested complexes of decorated block graphs.
\end{proposition}

\begin{proof}
  \vincent{todo}
\end{proof}

%%%%%%%%%%%%%%%%%%%%%%%%%%%%%%%%%%%%%%%
%%%%%%%%%%%%%%%%%%%%%%%%%%%%%%%%%%%%%%%

\section{Geometry: Spine Fans and Polytopes}

We denote by~$(\b{e}_v)_{v \in \vertexSet}$ the standard basis of~$\R^{\vertexSet}$.
We denote by~$\one_X \eqdef \sum_{x \in X} \b{e}_x$ the characteristic vector of a subset~$X \subseteq \vertexSet$, and write just~$\one$ for~$\one_{\vertexSet}$.
%We consider the hyperplane
%\[
%\Hyp \eqdef \bigset{\b{x} \in \R^{\vertexSet}}{\dotprod{\one}{\b{x}} = \weight_{\vertexSet}}.
%\]

%%%%%%%%%%%%%%%%%%%%%%%%%%%%%%%%%%%%%%%

\subsection{Spine fan}

%We first recall some observations on the braid arrangement (see \eg \cite{Postnikov, PostnikovReinerWilliams}) and on graphical arrangements.
%
Recall that a (polyhedral) \defn{cone} is the positive span of a finite set of vectors, or equivalently the intersection of a finite set of linear halfspaces.
The \defn{faces} of a cone are its intersections with its supporting linear hyperplanes.
A (polyhedral) \defn{fan}~$\c{F}$ is a collection of polyhedral cones closed under faces (if~$\polytope{C}$ is a face of~$\polytope{C'}$, then~$\polytope{C'} \in \c{F} \Rightarrow \polytope{C'} \in \c{F}$) and intersecting along faces (any two cones of~$\c{F}$ intersect along a face of both).
The \defn{rays} of~$\c{F}$ are its dimension~$1$ cones, and the \defn{regions} of~$\c{F}$ are its inclusion maximal cones.
The fan~$\c{F}$ is \defn{essential} when its inclusion minimal cone is the origin, \defn{complete} when its cones cover the space, and \defn{simplicial} when all its cones are simplicial (generated by linearly independent vectors).
In this paper, all the fans live in the hyperplane~${\Hyp[0] \eqdef \bigset{\b{x} \in \R^{\vertexSet}}{\dotprod{\one}{\b{x}} = 0}}$ of~$\R^{\vertexSet}$.

%We next consider two very classical examples of fans.
%
%\begin{example}
%  \label{exm:braidFan}
%  The \defn{braid arrangement} is the collection of hyperplanes~$\set{\b{x} \in \R^{\vertexSet}}{x_u = x_v}$ for all~$u \ne v \in \vertexSet$.
%  Since this arrangement is not essential (all its hyperplanes contain the line generated by~$\one$), we consider its intersection with the hyperplane~${\Hyp[0] \eqdef \bigset{\b{x} \in \R^{\vertexSet}}{\dotprod{\one}{\b{x}} = 0}}$.
%  This intersection defines the \defn{braid fan}~$\braidFan$.
%  Each surjection~$\pi : \vertexSet \to [k+1]$ corresponds to a $k$-dimensional cone~$\normalCone(\pi) \eqdef \set{\b{x} \in \Hyp[0]}{x_u \le x_v \text{ iff } \pi(u) \le \pi(v)}$ of~$\braidFan$.
%  In particular, the regions of~$\braidFan$ correspond to bijections~$\vertexSet \to [|\vertexSet|]$ and the rays of~$\braidFan$ correspond to proper subsets of~$\varnothing \ne X \ne \vertexSet$.
%  \vincent{Not sure if we want surjections or ordered partitions here.}
%%More generally, we associate to an acyclic oriented graph~$H$ whose nodes partition~$\vertexSet$ the cone~$\normalCone(H) \eqdef \set{\b{x} \in \Hyp[0]}{x_u \le x_v \text{ for all } u \preccurlyeq_H v}$, where~$u \preccurlyeq_H v$ means that there is a directed path in~$H$ from the node containing~$u$ to the node containing~$v$ (in particular, ~$u =_H v$ if~$u$ and~$v$ belong to the same node of~$H$).
%%Recall that a \defn{preposet} on~$\vertexSet$ is a reflexive and transitive binary relation~$R \subsetea \vertexSet \times \vertexSet$.
%%An equivalence relation is a symmetric preposet, and a poset if an antisymmetric preposet.
%%Any preposet~$R$ decomposes into an equivalence relation~$\equiv_R \eqdef \set{(u,v) \in R}{(v,u) \in R}$ together with a poset structure~${\prec_R \eqdef \, R/\!\equiv_R}$ on the equivalence classes of~$\equiv_R$.
%%Let~$H$ be an acyclic oriented graph~$H$ whose nodes partition~$\vertexSet$.
%%For two vertices~$u,v \in \vertexSet$, we write~$u \preccurlyeq_H v$ if there is a directed path in~$H$ from the node containing~$u$ to the node containing~$v$ (in particular, ~$u =_H v$ if~$u$ and~$v$ belong to the same node of~$H$).
%%We associate to~$H$ the two cones
%%\[
%%  \normalCone(H) \eqdef \set{\b{x} \in \Hyp[0]}{x_u \le x_v \text{ for all } u \preccurlyeq_H v}
%%  \quad\text{and}\quad
%%  \primalCone(H) \eqdef \cone\set{\b{e}_u - \b{e}_v}{\text{for all } u \preccurlyeq_H v}.
%%\]
%%Note that these two cones are polar to each other.
%\end{example}
%
%\vincent{We could introduce here the sylvester fan. But it would be a repetition with the spine fan. Not sure.}
%%\begin{example}
%%  \label{exm:sylvesterFan}
%%  \vincent{Todo}
%%\end{example}
%
%\begin{example}
%  \label{exm:graphicalFan}
%  The \defn{graphical arrangement} of~$G$ is the collections of hyperplanes~$\set{\b{x} \in \R^{\vertexSet}}{x_u = x_v}$ for all~$\{u,v\} \in \edgeSet$.
%  Its intersection with the hyperplane~$\Hyp[0]$ defines the \defn{graphical fan}~$\graphicalFan$.
%  The cones of~$\graphicalFan$ correspond to the pairs~$(p,o)$ where~$p$ is a partition of~$\vertexSet$ into connected subgraphs, and~$o$ is an acyclic orientation of the quotient graph~$G/p$.
%  In particular, the regions of~$\graphicalFan$ correspond to the acyclic orientations of~$G$.
%\end{example}
%
%We consider the following fan which is sandwiched between the braid fan and the graphical fan.

\begin{definition}
  \label{def:spineFan}
  For a spine~$\spine$, we denote by~$\normalCone(\spine)$ the cone of~$\Hyp[0]$ defined by the inequalities~${x_u \le x_v}$ for all directed paths joining~$u$ to~$v$ in~$\spine$ (in particular, $x_u = x_v$ when $u,v$ belong to the same node of~$\spine$).
%  \[
%    \normalCone(\spine) \eqdef \set{\b{x} \in \Hyp[0]}{x_u \le x_v \text{ for all (possibly trivial) directed paths } u \to v \text{ in } \spine}
%  \]
  The \defn{spine fan}~$\spineFan$ is the collection of cones~$\normalCone(\spine)$ for all spines~$\spine$ on~$\b{G}$.  
\end{definition}

\begin{proposition}
  \label{prop:spineFan}
  The spine fan~$\spineFan$ is a complete simplicial fan on~$\Hyp[0]$.
\end{proposition}

\begin{proof}
  Since each spine~$\spine$ on~$\b{G}$ is a directed tree, the cone~$\normalCone(\spine)$ is simplicial, and its faces are the cones~$\normalCone(\spine')$ for all spines~$\spine'$ obtained by repeated arc contractions in~$\spine$ (in other words, for all spines in the lower ideal of the spine poset~$\spinePoset$ generated by~$\spine$).
  We thus just need to prove that the cones~$\normalCone(\spine)$ properly intersect and cover~$\Hyp[0]$, or equivalently, that the relative interiors of the cones~$\normalCone(\spine)$ partition~$\Hyp[0]$.
  This immediately follows from \cref{subsec:surjectionMaps}.
  \vincent{Be more precise here.}
\end{proof}

\begin{corollary} 
   The nested complex~$\nestedComplex$ is a simplicial sphere.
\end{corollary}

%\begin{proposition}
%  \label{prop:fanSandwich}
%  The spine fan~$\spineFan$ coarsens the braid fan~$\braidFan$ and refines the graphical fan~$\graphicalFan$.
%\end{proposition}
%
%\begin{proof}
%  This immediately follows from \cref{subsec:surjectionMaps}.
%\end{proof}

%\begin{example}
%  \label{exm:braidFanGraphicalFan}
%  The spine fan~$\spineFan$ recovers the braid fan~$\braidFan$ of \cref{exm:braidFan} and the graphical fan~$\graphicalFan$ of \cref{exm:graphicalFan} in the following two extreme situations:
%  \begin{itemize}
%    \item if~$\Down[\vertexSet] = \Up[\vertexSet] = \varnothing$, then the spine fan~$\spineFan$ is the braid fan~$\braidFan$,
%    \item if~$\Down[\vertexSet] = \Up[\vertexSet] = \vertexSet$, then the spine fan~$\spineFan$ is the graphical fan~$\graphicalFan$.
%  \end{itemize}
%\end{example}

\begin{example}
  \label{exm:spineFans}
  For instance, the spine fan is:
  \begin{enumerate}[(i)]
    \item the \defn{braid fan}~$\braidFan$ when~$\b{G}$ is complete or undecorated. It is the fan defined from the classical \defn{braid arrangement} of the hyperplanes~$\set{\b{x} \in \R^{\vertexSet}}{x_u = x_v}$ for all vertices~$u \ne v \in \vertexSet$. It has a $k$-dimensional cone~$\normalCone(\pi) \eqdef \set{\b{x} \in \Hyp[0]}{x_u \le x_v \text{ iff } \pi(u) \le \pi(v)}$ for each surjection~$\pi : \vertexSet \to [k+1]$. In particular, the regions of~$\braidFan$ correspond to bijections~${\vertexSet \to [|\vertexSet|]}$ and the rays of~$\braidFan$ correspond to proper subsets of~$\varnothing \ne X \ne \vertexSet$.
    \item the \defn{nested fan} of~$G$ studied in~\cite{CarrDevadoss, Zelevinsky} when~$\b{G}$ is down decorated (in particular the classical \defn{sylvester fan} when~$\b{G}$ is a down decorated path).
    \item the \defn{graphical fan}~$\graphicalFan$ when~$\b{G}$ is fully decorated (in particular the \defn{coordinate fan} when~$\b{G}$ is a fully decorated tree). It is the fan defined from the \defn{graphical arrangement} of the hyperplanes~$\set{\b{x} \in \R^{\vertexSet}}{x_u = x_v}$ for all edges~$\{u, v\} \in \vertexSet$. The cones of~$\graphicalFan$ correspond to the pairs~$(p,o)$ where~$p$ is a partition of~$\vertexSet$ into connected subgraphs, and~$o$ is an acyclic orientation of the quotient graph~$G/p$. In particular, the regions of~$\graphicalFan$ correspond to the acyclic orientations of~$G$.
    \item the various \defn{permutree fans} of~\cite{PilaudPons-permutrees} when~$G$ is a path (in particular the various type~$A$ \defn{Cambrian fans} for up-down decorated paths).
  \end{enumerate}
\end{example}

\begin{proposition}
  \label{prop:refinementSpineFan}
  If~$\b{G}$ refines~$\b{G}'$, the spine fan~$\spineFan[\b{G}]$ refines the spine fan~$\spineFan[\b{G}']$.
  In particular, the spine fan~$\spineFan$ coarsens the braid fan~$\braidFan$ and refines the graphical fan~$\graphicalFan$.
\end{proposition}

\begin{proof}
  Follows from \cref{prop:refinementEquivalenceRelations}.
\end{proof}

%%%%%%%%%%%%%%%%%%%%%%%%%%%%%%%%%%%%%%%

\subsection{Spine polytope}

Recall that a \defn{polytope}~$\polytope{P}$ is the convex hull of a finite set of points, or equivalently a bounded intersection of a finite set of affine halfspaces.
The \defn{faces} of~$\polytope{P}$ are its intersections with its supporting affine hyperplanes.
The \defn{vertices} of~$\polytope{P}$ are its dimension~$0$ faces, and the \defn{facets} of~$\polytope{P}$ are its codimension~$1$ faces.
The \defn{normal cone} of a face~$\polytope{F}$ of~$\polytope{P}$ is the cone of all linear functions whose maximum over~$\polytope{P}$ is attained precisely on~$\polytope{F}$.
The \defn{normal fan} of~$\polytope{P}$ is the fan formed by the normal cones of all the faces of~$\polytope{P}$.
All our polytopes are parametrized by a weight function, which does not change their combinatorics but provides more freedom for their geometry.

\begin{definition}
  \label{def:weight}
  For a set~$X$, we denote by~$\monombinom{X}$ the set of~$\binom{|X|+1}{2}$ subsets of~$X$ of size $1$ or~$2$.
  We fix a \defn{weight}~$\weight_p$ for all~$p \in \monombinom{\vertexSet}$, and we define~$\weight_X \eqdef \sum_{p \in \monombinom{X}} \weight_p$ for any subset~$X$ of~$\vertexSet$.
\end{definition}

%We next consider two very classical examples of polytopes.
%
%\begin{example}
%  \label{exm:permutahedron}
%  The \defn{permutahedron}~$\Perm$ is the polytope defined equivalently as
%  \begin{itemize}
%%    \item the convex hull of the points~$\sum_{u,v \in \vertexSet} \delta_{\sigma(u) \le \sigma(v)} \weight_{uv} \b{e}_v$ for all bijections~$\sigma: \vertexSet \to [|\vertexSet|]$,
%    \item the convex hull of the points~$\point[\sigma] \eqdef \sum_{p \in \monombinom{\vertexSet}} \weight_p \b{e}_{\max_\sigma(p)}$ for all bijections~$\sigma: \vertexSet \to [|\vertexSet|]$,
%    \item the intersection of the affine hyperplane~$\Hyp \eqdef \bigset{\b{x} \in \R^{\vertexSet}}{\dotprod{\one}{\b{x}} = \weight_{\vertexSet}}$ with the affine halfspaces~$\HS_X \eqdef \bigset{\b{x} \in \R^{\vertexSet}}{\dotprod{\one_X}{\b{x}} \ge \weight_X}$ for all proper subsets~$\varnothing \ne X \subsetneq \vertexSet$,
%    \item the Minkowski sum~$\sum_{p \in \monombinom{\vertexSet}} \weight_p \triangle_p$, where~$\triangle_p \eqdef \conv\set{\b{e}_v}{v \in p}$.
%  \end{itemize}
%  The braid fan~$\braidFan$ is the normal fan of the permutahedron~$\Perm$.
%  Note that the barycenter of~$\Perm$ is the point~$\b{0}^\weight \eqdef \sum_{p \in \monombinom{\vertexSet}} \frac{\weight_p}{|p|} \one_p$.
%\end{example}
%
%\vincent{We could introduce here the graph associahedron. But it would be a repetition with the spine fan. Not sure.}
%\begin{example}
%  \label{exm:associahedron}
%  The \defn{graph associahedron}~$\Asso$ is the polytope defined equivalently as
%  \vincent{todo}
%  \begin{itemize}
%    \item the convex hull of the points~$\point[{\spine[T]}] \eqdef \dots$,
%    \item the intersection of the affine hyperplane~$\Hyp \eqdef \bigset{\b{x} \in \R^{\vertexSet}}{\dotprod{\one}{\b{x}} = \weight_{\vertexSet}}$ with the affine halfspaces~$\HS_X \eqdef \bigset{\b{x} \in \R^{\vertexSet}}{\dotprod{\one_X}{\b{x}} \ge \weight_X}$ for all proper subsets~$\varnothing \ne X \subsetneq \vertexSet$ which induce a connected subgraph of~$G$,
%    \item the Minkowski sum~$\sum_{X} \weight_X \triangle_X$, where~$\triangle_X \eqdef \conv\set{\b{e}_u}{u \in X}$.
%  \end{itemize}
%  The sylvester fan~$\sylvesterFan$ is the normal fan of the associahedron~$\Asso$.
%\end{example}
%
%\begin{example}
%  \label{exm:graphicalZonotope}
%  The \defn{graphical zonotope}~$\Zono$ is the polytope defined as...
%  The graphical fan~$\graphicalFan$ is the normal fan of the graphical zonotope~$\Zono$.
%  \vincent{todo. Be careful with the translation part...}
%\end{example}
%
%We consider the following polytope which is sandwiched between the permutahedron and the graphical zonotope.

\begin{definition}
  For a maximal spine~$\spine$ on~$\b{G}$, we denote by~$\Pi(\spine)$ the set of (undirected and simple) paths joining two (possibly identical) nodes of~$\spine$.
  For a path~$\pi \in \Pi(\spine)$, we denote by
  \begin{itemize}
    \item $\boundary$ the set of endpoints of~$\pi$, % and $\weight_\pi$ the weight~$\weight_{\boundary}$, 
    \item $\peaks$ (resp.~$\valleys$) the set of \defn{peaks} (resp.~\defn{valleys}) of~$\pi$, \ie nodes~$u$ of~$\pi$ where no outgoing (resp. incoming) arc of~$u$ is used by~$\pi$.
  \end{itemize}
\end{definition}

\begin{theorem}
  \label{thm:permutreehedra}
  For any weight matrix~$\weight$, the spine fan~$\spineFan$ is the normal fan of the \defn{spine polytope}~$\Spin$ defined irredundantly and equivalently as
  \begin{enumerate}[(i)]
    \item the convex hull of the points~$\point \eqdef \sum_{\pi \in \Pi(\spine)} \weight_{\boundary} \big( \one_{\peaks} - \one_{\valleys \ssm \boundary} \big)$ for all corank~$0$ spines~$\spine$~on~$\b{G}$,
    \item the intersection of the affine hyperplane~$\Hyp \eqdef \bigset{\b{x} \in \R^{\vertexSet}}{\dotprod{\one}{\b{x}} = \weight_{\vertexSet}}$ with the affine halfspaces~$\HS_B \eqdef \bigset{\b{x} \in \R^{\vertexSet}}{\dotprod{\one_B}{\b{x}} \ge \weight_B}$ for all blocks of~$\b{G}$.
%    \item the intersection of the hyperplane~$\Hyp$ with the halfspaces~$\HS_B$ for all blocks of~$\b{G}$.
  \end{enumerate}
\end{theorem}

Our proof of \cref{thm:permutreehedra} is based on the following result of C.~Hohlweg, C.~Lange, and H.~Thomas~\cite{HohlwegLangeThomas}.

\begin{theorem}[\protect{\cite[Thm.~4.1]{HohlwegLangeThomas}}]
  \label{thm:HohlwegLangeThomas}
  Given an essential complete simplicial fan~$\c{F}$ in~$\R^d$, consider a point~$\b{a}_C$ of~$\R^d$ for each chamber~$C$ of~$\c{F}$ and a halfspace~$\HS[]_\rho$ of~$\R^d$ for each ray~$\rho$ of~$\c{F}$, containing the origin and supported by a hyperplane~$\Hyp[]_\rho$ orthogonal to~$\rho$.
  If 
  \begin{itemize}
%    \item for each ray~$\rho$, the halfspace~$\HS[]_\rho$ contains the origin and is supported by a hyperplane~$\Hyp[]_\rho$ orthogonal to~$\rho$,
    \item for each ray~$\rho$ in a chamber~$C$, the point $\b{a}_C$ is contained in the hyperplane~$\Hyp[]_\rho$, and
    \item for each pair~$C,C'$ of adjacent chambers of~$\c{F}$, the vector~$\b{a}_C - \b{a}_{C'}$ points from~$C'$ to~$C$,
  \end{itemize}
  then the descriptions
  \[
    \conv\set{\b{a}_C}{C \text{ chamber of } \c{F}}
    \qquad\text{and}\qquad
    \bigcap_{\rho \text{ ray of } \c{F}} \HS[]_\rho
  \]
  are irredundant and define the same polytope whose normal fan is~$\c{F}$.
  \qed
\end{theorem}

Our next two lemmas prove that the two conditions of \cref{thm:HohlwegLangeThomas} are fulfilled by the points and halspaces of \cref{thm:permutreehedra}.
\vincent{discussion about the origin here...}

\begin{lemma}
  \label{lem:vertexFacetIncidences}
  We have~$\dotprod{\one}{\point} = \weight_{\vertexSet}$ and $\dotprod{\one_{\so[\gamma]}}{\point} = \weight_{\so[\gamma]}$ for all arcs~$\gamma$ of~$\spine$.
\end{lemma}

\begin{proof}
  Along any path~$\pi$, the peaks and valleys alternate, so that~$\dotprod{\one}{\one_{\peaks} - \one_{\valleys \ssm \boundary}} = 1$.
  This shows that~$\dotprod{\one}{\point} = \sum_{\pi \in \Pi(\spine)} \weight_{\boundary} = \weight_{\vertexSet}$ since any two vertices of~$\vertexSet$ are connected by precisely one simple path in~$\spine$.
  
  Consider now an arc~$\gamma$ of~$\spine$.
  Again, the peaks and valleys alternate along any path~$\pi$.
  Moreover, if~$\pi$ contains~$\gamma$, then the closest node to~$\gamma$ in~$\peaks \cup \valleys$ in~$\so[\gamma]$ (resp.~in~$\ta[\gamma]$) is a valley (resp.~a peak).
  Hence, $\dotprod{\one_{\so[\gamma]}}{\one_{\peaks} - \one_{\valleys \ssm \boundary}} = 1$ if~$\gamma$ is included in~$\so[\gamma]$ and~$0$ otherwise.
  We conclude that~$\dotprod{\one_{\so[\gamma]}}{\point} = \sum_{\pi \in \Pi(\spine), \; \boundary \subseteq \so[\gamma]} \weight_{\boundary} = \weight_{\vertexSet}$ since any two vertices of~$\so[\gamma]$ are connected by precisely one simple path in~$\spine$.
\end{proof}

\begin{lemma}
  \label{lem:flipDifference}
  If~$\spine$ and~$\spine'$ are two corank~$0$ spines on~$\b{G}$ such that~$\spine'$ is obtained from~$\spine$ by flipping an arc joining node~$x$ to node~$y$, then the difference~$\point[\spine] - \point[\spine']$ is a positive multiple of~$\b{e}_x - \b{e}_y$.
\end{lemma}

\begin{proof}
  Let~$A_x$ (resp.~$B_x$) denote the union of the source (resp.~target) sets of the arcs which are incoming to~$x$ (resp.~outgoing from~$x$) in both~$\spine$ and~$\spine'$, and define similarly~$A_y$ (resp.~$B_y$).
  Let~$u,v \in \vertexSet$ and denote by~$\pi$ (resp.~$\pi'$) the path in~$\spine$ (resp.~$\spine'$) joining~$u$ to~$v$.
  Observe that the paths~$\pi$ and~$\pi'$ have the same peaks and valleys, except if
  \begin{itemize}
    \item $u \in A_x$ and~$v \in \{y\} \cup A_y$, then $x \in \peaks[\pi'] \ssm \peaks[\pi]$ while~$y \in \peaks[\pi] \ssm \peaks[\pi']$,
    \item $u \in A_x$ and~$v \in B_y$, then $x \in \peaks[\pi'] \ssm \peaks[\pi]$ while~$y \in \valleys[\pi'] \ssm \valleys[\pi]$,
    \item $u \in \{x\} \cup B_x$ and~$v \in \{y\} \cup A_y$, then $x \in \valleys[\pi] \ssm \valleys[\pi']$ while~$y \in \peaks[\pi] \ssm \peaks[\pi']$,
    \item $u \in \{x\} \cup B_x$ and~$v \in B_y$, then $x \in \valleys[\pi] \ssm \valleys[\pi']$ while~$y \in \valleys[\pi'] \ssm \valleys[\pi]$.
  \end{itemize}
  We thus obtain that
  \(
    \point[\spine] - \point[\spine'] = \weight_{X \times Y} (\b{e}_x - \b{e}_y)
  \)
  where~$\weight_{X \times Y}$ denotes the sum of the weights of the pairs formed by an element of~$X \eqdef \{x\} \cup A_x \cup B_x$ and an element of~$Y \eqdef \{y\} \cup A_y \cup B_y$.
  Finally, we observe that~$\weight_{X \times Y} \ge \weight_{\{x,v\}} > 0$ by our positivity assumption on the weight function.
\end{proof}

\begin{proof}[Proof of \cref{thm:permutreehedra}]
  For the spine polytope of \cref{thm:permutreehedra}, the conditions of \cref{thm:HohlwegLangeThomas} are guarantied by~\cref{lem:vertexFacetIncidences,lem:flipDifference}.
  It follows that the spine polytope~$\Spin$ realizes the nested complex~$\nestedComplex$ and that its normal fan is the spine fan~$\spineFan$.
\end{proof}

%\begin{example}
%  \label{exm:permutahedronGraphicalZonotope}
%  The spine polytope~$\Spin$ recovers the permutahedron~$\Perm$ of \cref{exm:permutahedron} and the graphical zonotope~$\Zono$ of \cref{exm:graphicalZonotope} in the following two extreme situations:
%  \begin{itemize}
%    \item if~$\Down[\vertexSet] = \Up[\vertexSet] = \varnothing$, then the spine polytope~$\Spin$ is the permutahedron~$\Perm$,
%    \item if~$\Down[\vertexSet] = \Up[\vertexSet] = \vertexSet$, then the spine polytope~$\Spin$ is the graphical zonotope~$\Zono$.
%  \end{itemize}
%\end{example}

\begin{example}
  \label{exm:spinePolytopes}
  For instance, the spine polytope is:
  \begin{enumerate}[(i)]
    \item the classical \defn{permutahedron}~$\Perm$ when~$\b{G}$ is complete or undecorated.
    \item the \defn{graph associahedron} of~$G$ studied in~\cite{CarrDevadoss, Devadoss, Postnikov, FeichtnerSturmfels, Zelevinsky, Pilaud-removahedra} when~$\b{G}$ is down decorated (in particular the classical \defn{associahedron} of~\cite{ShniderSternberg, Loday} when~$\b{G}$ is a down decorated~path).
    \item the \defn{graphical zonotope}~$\Zono$ when~$\b{G}$ is fully decorated (in particular a \defn{parallelepiped} when~$\b{G}$ is a fully decorated tree).
    \item the various \defn{permutreehedra} of~\cite{PilaudPons-permutrees} when~$G$ is a path (in particular the various \defn{associahedra} of~\cite{HohlwegLange, LangePilaud} for up-down decorated paths).
  \end{enumerate}
\vincent{toy example and low-dimensional examples.}
\end{example}

\begin{proposition}
  \label{prop:refinementSpinePolytopes}
  If~$\b{G}$ refines~$\b{G}'$, then all facet inequalities of~$\Spin[\b{G}]$ are facet inequalities of~$\Spin[\b{G}']$.
  In particular, the spine polytope~$\Spin$ contains the permutahedron~$\Perm$ and is contained in the graphical zonotope~$\Zono$.
\end{proposition}

\begin{proof}
  \vincent{todo}
\end{proof}

%%%%%%%%%%%%%%%%%%%%%%%%%%%%%%%%%%%%%%%

\subsection{Further geometric properties}

\subsubsection{Faces}

\begin{proposition}
  \label{prop:faces}
  Any face of the spine polytope~$\Spin$ is a Cartesian product of spine polytopes.
  \vincent{Is it true here of only in the next section?}
\end{proposition}

\begin{proof}
  \vincent{todo}
\end{proof}

\subsubsection{Spine poset as orientation of the spine polytope}

\subsubsection{Parallel facets}

\subsubsection{Common vertices}

\subsubsection{Isometry classes}

%%%%%%%%%%%%%%%%%%%%%%%%%%%%%%%%%%%%%%%

\subsection{Loday's weights}

%%%%%%%%%%%%%%%%%%%%%%%%%%%%%%%%%%%%%%%
%%%%%%%%%%%%%%%%%%%%%%%%%%%%%%%%%%%%%%%

\section{Algebra: operad structure}

%%%%%%%%%%%%%%%%%%%%%%%%%%%%%%%%%%%%%%%
%%%%%%%%%%%%%%%%%%%%%%%%%%%%%%%%%%%%%%%

\appendix

\section{More combinatorics}

%%%%%%%%%%%%%%%%%%%%%%%%%%%%%%%%%%%%%%%

\subsection{Nestings, tubings and Forcey-Ronco substitution on graph-associahedra}

%%%%%%%%%%%%%%%%%%%%%%%%%%%%%%%%%%%%%%%

\subsection{Generalized tubings}

\vincent{I am not sure that this section will stay in the final version.}

Let~$M$ be a maple tree and~$G = (V,E)$ be the block graph obtained by tapping~$M$.
There are four combinatorial models for the facets of a block graph permutreehedron:
\begin{enumerate}
  \item a \defn{rank~$1$ spine} on~$G$, \ie a spine on~$G$ with precisely two nodes,
  \item a \defn{maple subtree} of~$M$, \ie a maple tree induced by a connected subset of~$M$, or equivalently a connected component of $M \ssm B$ for a subset~$B$ of blue vertices of~$M$,
  \item a \defn{signed tube} of~$G$, \ie a pair~$(\Down[Z], \Up[Z])$ of maple subtrees of~$M$ such that $\partial \Down[Z] \subseteq \Down[Y] \cap \Up[Z]$ and $\partial \Up[Z] \subseteq \Up[Y] \cap \Down[Z]$,
  \item a \defn{signed biconvex subset} of~$G$, \ie a subset~$Y$ of vertices of~$G$ which is positive convex and whose complement is negative convex. This is also what we called splittable partition.
\end{enumerate}
We can read the spine complex on these 4 models.









\section*{Acknowledgments}

CRM in Barcelona 

\bibliographystyle{alpha}
\bibliography{blockgraph}
\label{sec:biblio}

\end{document}