\documentclass{amsart}

\usepackage[T1]{fontenc}
\usepackage{%enumerate, 
amsmath, amsfonts, amssymb, amsthm, mathrsfs, wasysym, graphics, graphicx, xcolor, url, hyperref, hypcap, xargs, multicol, pdflscape, multirow, hvfloat, array, ae, aecompl, pifont, mathtools, a4wide, float, blkarray, overpic, nicefrac}
\usepackage[shortlabels, inline]{enumitem}%shortlabels to have same syntax as enumerate package, inline gives inline option with *
\usepackage{bbm}%allows for \mathbbm{1}
\usepackage[noabbrev,capitalise]{cleveref}
\usepackage[normalem]{ulem}
\usepackage{marginnote}
\hypersetup{colorlinks=true, citecolor=darkblue, linkcolor=darkblue}
\usepackage[all]{xy}
\usepackage{tikz}
\usepackage{tikz-cd}
%\usepackage{tkz-graph}
\usetikzlibrary{trees, decorations, decorations.pathmorphing, decorations.markings, decorations.shapes, shapes, arrows, matrix, calc, fit, intersections, patterns, angles}
\graphicspath{{figures/}{figures/diagonals/}{figures/walks/}{figures/tubes/}{figures/blocks/}}
\makeatletter\def\input@path{{figures/}}\makeatother
\usepackage{caption}
\captionsetup{width=\textwidth}
\usepackage[export]{adjustbox}

%%%%%%%%%%%%%%%%%%%%%%%%%%%%%%%%%%%%%%

% theorems
\newtheorem{theorem}{Theorem}[section]
\newtheorem{corollary}[theorem]{Corollary}
\newtheorem{proposition}[theorem]{Proposition}
\newtheorem{lemma}[theorem]{Lemma}
\newtheorem{conjecture}[theorem]{Conjecture}
\newtheorem*{theorem*}{Theorem}%[section]

\theoremstyle{definition}
\newtheorem{definition}[theorem]{Definition}
\newtheorem{example}[theorem]{Example}
\newtheorem{remark}[theorem]{Remark}
\newtheorem{question}[theorem]{Question}
\newtheorem{notation}[theorem]{Notation}
\newtheorem{assumption}[theorem]{Assumption}
\newtheorem{convention}[theorem]{Convention}

\crefname{equation}{Equation}{Equations}

% math special letters
\newcommand{\R}{\mathbb{R}} % reals
\newcommand{\Q}{\mathbb{Q}} % rationals
\newcommand{\N}{\mathbb{N}} % naturals
\newcommand{\Z}{\mathbb{Z}} % integers
\newcommand{\C}{\mathbb{C}} % complex
\newcommand{\I}{\mathbb{I}} % set of integers
\newcommand{\HH}{\mathbb{H}} % hyperplane
\newcommand{\K}{k} % field
\newcommand{\bb}[1]{{\mathbb{#1}}} % mathbb letters
\newcommand{\f}[1]{{\mathfrak{#1}}} % mathfrak letters
\renewcommand{\c}[1]{{\mathcal{#1}}} % call letters
\renewcommand{\b}[1]{{\boldsymbol{#1}}} % bold letters
\newcommand{\h}{\widehat} % hat letters

% math commands
\newcommand{\set}[2]{\left\{ #1 \;\middle|\; #2 \right\}} % set notation
\newcommand{\bigset}[2]{\big\{ #1 \;\big|\; #2 \big\}} % big set notation
\newcommand{\Bigset}[2]{\Big\{ #1 \;\Big|\; #2 \Big\}} % Big set notation
\newcommand{\setangle}[2]{\left\langle #1 \;\middle|\; #2 \right\rangle} % set notation
\newcommand{\ssm}{\smallsetminus} % small set minus
\newcommand{\dotprod}[2]{\langle \, #1 \; | \; #2 \, \rangle} % dot product
\newcommand{\bigdotprod}[2]{\big\langle \, #1 \; \big| \; #2 \, \big\rangle} % dot product
\newcommand{\symdif}{\,\triangle\,} % symmetric difference
\newcommand{\one}{\mathbbm{1}} % the all one vector
\newcommandx{\ones}[1][1=n]{\one_{#1}} % the all one vector of length n
\newcommand{\eqdef}{\mbox{\,\raisebox{0.2ex}{\scriptsize\ensuremath{\mathrm:}}\ensuremath{=}\,}} % :=
\newcommand{\defeq}{\mbox{~\ensuremath{=}\raisebox{0.2ex}{\scriptsize\ensuremath{\mathrm:}} }} % =:
\newcommand{\simplex}{\triangle} % simplex
\renewcommand{\implies}{\Rightarrow} % imply sign
\newcommand{\transpose}[1]{{#1}^T} % transpose matrix
\newcommand{\truth}[1]{\left[ #1 \right]} % truth (kronecker delta)

% operators
\DeclareMathOperator{\conv}{conv} % convex hull
\DeclareMathOperator{\vect}{vect} % linear span
\DeclareMathOperator{\cone}{cone} % cone hull

% others
\newcommand{\ie}{\textit{i.e.}~} % id est
\newcommand{\eg}{\textit{e.g.}~} % exempli gratia
\newcommand{\Eg}{\textit{E.g.}~} % exempli gratia
\newcommand{\apriori}{\textit{a priori}} % a priori
\newcommand{\viceversa}{\textit{vice versa}} % vice versa
\newcommand{\versus}{\textit{vs.}~} % versus
\newcommand{\aka}{\textit{a.k.a.}~} % also known as
\newcommand{\perse}{\textit{per se}} % per se
\newcommand{\ordinal}{\textsuperscript{th}} % th for ordinals
\newcommand{\ordinalst}{\textsuperscript{st}} % st for ordinals
\definecolor{darkblue}{rgb}{0,0,0.7} % darkblue color
\definecolor{green}{RGB}{57,181,74} % green color
\definecolor{violet}{RGB}{147,39,143} % violet color
\newcommand{\red}{\color{red}} % red command
\newcommand{\blue}{\color{blue}} % blue command
\newcommand{\orange}{\color{orange}} % orange command
\newcommand{\green}{\color{green}} % green command
\newcommand{\darkblue}{\color{darkblue}} % darkblue command
\newcommand{\defn}[1]{\textsl{\darkblue #1}} % emphasis of a definition
\newcommand{\para}[1]{\medskip\noindent\uline{\textit{#1.}}} % paragraph
\renewcommand{\topfraction}{1} % possibility to have one page of pictures
\renewcommand{\bottomfraction}{1} % possibility to have one page of pictures
\newcommand{\ex}{_{\textrm{exm}}} % examples
\newcommand{\pa}{_{\textrm{pa}}} % path
\newcommand*\circled[1]{\tikz[baseline=(char.base)]{\node[shape=circle, draw, inner sep=1.5pt, scale=.7] (char) {#1};}}
\newcommand{\compactVectorD}[2]{\begin{bmatrix} #1 \\ #2 \end{bmatrix}}
\newcommand{\compactVectorT}[3]{\begin{bmatrix} #1 \\[-.1cm] #2 \\[-.1cm] #3 \end{bmatrix}}

% marginal comments
\usepackage{todonotes}
\newcommand{\guillaume}[1]{\todo[color=orange!30]{#1 --- G.}}
\newcommand{\vincent}[1]{\todo[color=blue!30]{#1 \\ \hfill --- V.}}

% polytopes
\newcommand{\polytope}[1]{\mathsf{#1}} % font polytopes
\newcommandx{\Perm}[1][1=n]{\polytope{Perm}_{#1}} % permutahedron

% geometry
\newcommandx{\Asso}[2][1=n,2={}]{\mathsf{Asso}^{#2}(#1)} % associahedron
\newcommandx{\Nest}[2][1=\building,2={}]{\mathsf{Nest}^{#2}(#1)} % associahedron
% \newcommandx{\Zono}[2][1=n,2={}]{\mathsf{Zono}^{#2}(#1)} % zonotope
\newcommand{\walls}{\b{W}} % walls

\newcommandx{\Fan}[1][1=F]{\mathcal{#1}} % fan
\newcommand{\multiplicityVector}{\b{m}} % multiplicity vector
\newcommand{\gvector}[1]{\b{g}(#1)} % g-vector of #1
\newcommand{\gvectorFull}[2]{\b{g}(#1,#2)} % g-vector of #2 wrt #1
\newcommand{\gvectors}[1]{\b{g}(#1)} % g-vectors of #1
\newcommand{\gvectorsFull}[2]{\b{g}(#1,#2)} % g-vectors of #2 wrt #1
\newcommandx{\nestedFan}[1][1=\quiver]{\mathcal{F}(#1)} % g-vector fan
\newcommand{\cvector}[2]{\mathbf{c}(#2 \in #1)} % c-vector of the cluster variable #2 in the cluster #1
\newcommand{\cvectorFull}[3]{\mathbf{c}(#1,#3 \in #2)} % c-vector of the cluster variable #3 in the cluster #2 with respect to the initial cluster #1
\newcommand{\cvectors}[1]{\mathbf{c}(#1)} % c-vectors of the cluster #1
\newcommand{\cvectorsFull}[2]{\mathbf{c}(#1,#2)} % c-vectors of the cluster #2 with respect to the initial cluster #1
\newcommand{\ivector}[1]{\b{\iota}_{#1}} % i-vector of #1
\newcommandx{\ray}[1][1=r]{\b{#1}} % ray
\newcommandx{\rays}[1][1=R]{\b{#1}} % rays

% graphical zonotopes
\newcommandx{\gZono}[1][1=G]{\mathsf{Z}_{#1}} % graphical zonotope
\newcommandx{\gArr}[1][1=G]{\mathcal{A}_{#1}} % graphical arrangement
\newcommandx{\gFan}[1][1=G]{\Fan_{#1}} % graphical fan projected
\newcommandx{\gFanO}[1][1=G]{\widehat{\Fan}_{#1}} % graphical fan unprojected
\newcommandx{\cc}[1][1=G]{\mathbb{K}_{#1}} % connected components space
\newcommandx{\braid}[1][1=n]{\mathcal{B}_{#1}} %braid fan/arrangement
\newcommandx{\sbraid}[1][1=n]{\widehat{\mathcal{B}}_{#1}} %split braid fan
\newcommandx{\dZono}[1][1=\b{h}]{\mathsf{D}_{#1}} % h-deformed zonotope



%Decorations
\newcommandx{\up}[1][1=n]{\overline{#1}} 
\newcommandx{\down}[1][1=n]{\underline{#1}} 
\newcommandx{\updown}[1][1=n]{\overline{\underline{#1}}} 

%Source and target
\newcommandx{\so}[1][1=i]{\mathrm{s}(#1)} 
\newcommandx{\ta}[1][1=o]{\mathrm{t}(#1)} 

%Set of spines
\newcommandx{\spines}[1][1=G]{\mathcal{S}(#1)} 
\newcommandx{\maxspines}[1][1=G]{\mathcal{MS}(#1)} 

% Type cone
\newcommand{\deformationCone}{\mathbb{DC}} % deformation cone
\newcommand{\typeCone}{\mathbb{TC}} % type cone
\newcommand{\ctypeCone}{\smash{\overline{\mathbb{TC}}}} % type cone
\newcommandx{\coefficient}[3][1={\b{s}}, 2=\b{r}, 3=\b{r}']{\alpha_{#2,#3}(#1)} % coefficient in linear dependence
\newcommandx{\virtualPolytopes}[1][1=d]{\mathbb{V}^{#1}} % virtual polytopes
\newcommandx{\VDP}[1][1=n]{\mathbb{VDP}^{#1}} % virtual deformed permutahedra
\newcommandx{\CVDP}[1][1=n]{\overrightarrow{\mathbb{VDP}}^{#1}} % caged virtual deformed permutahedra
\newcommand{\VD}[1][1=n]{\mathbb{VD}} % virtual deformations
\newcommand{\Weight}{\mathbb{W}} % 1st weight space
\newcommandx{\opcone}[1][1={\mu,\omega}]{\polytope{C}_{#1}}
\newcommandx{\orcone}[1][1={\omega}]{\polytope{C}_{#1}}

% formating the part command
\makeatletter
\def\part{\@startsection{part}{1}%
\z@{.7\linespacing\@plus\linespacing}{.8\linespacing}%
{\LARGE\sffamily\centering}}
%\@addtoreset{section}{part}
\makeatother
\renewcommand{\thepart}{\Roman{part}}
%\renewcommand{\thesection}{\arabic{part}.\arabic{section}}

% formating the table of contents
\setcounter{tocdepth}{4}
\makeatletter
\def\l@section{\@tocline{1}{5pt}{0pc}{}{}}
\makeatother
\let\oldtocpart=\tocpart
\renewcommand{\tocpart}[2]{\sc\large\oldtocpart{#1}{#2}}
\let\oldtocsection=\tocsection
\renewcommand{\tocsection}[2]{\bf\oldtocsection{#1}{#2}}
\let\oldtocsubsubsection=\tocsubsubsection
\renewcommand{\tocsubsubsection}[2]{\quad\oldtocsubsubsection{#1}{#2}}

%Drapeau européen

\usepackage{graphicx,calc}
\newlength\myheight
\newlength\mydepth
\settototalheight\myheight{Xygp}
\settodepth\mydepth{Xygp}
\setlength\fboxsep{0pt}
\newcommand*\inlinegraphics[1]{%
  \settototalheight\myheight{Xygp}%
  \settodepth\mydepth{Xygp}%
  \raisebox{-\mydepth}{\includegraphics[height=\myheight]{#1}}%
}

%%%%%%%%%%%%%%%%%%%%%%%%%%%%%%%%%%%%%%

\title{Block graph permutreehedra}

\author{Guillaume Laplante-Anfossi}
\address[Guillaume Laplante-Anfossi]{Universit\'e Sorbonne Paris Nord, Laboratoire Analyse, G\'eom\'etrie et Applications, CNRS, UMR 7539, F-93430 Villetaneuse, France}
\email{laplante-anfossi@math.univ-paris13.fr}
\urladdr{\url{https://www.math.univ-paris13.fr/~laplante-anfossi/}}

\author{Vincent Pilaud}
\address[Vincent Pilaud]{CNRS \& LIX, \'Ecole Polytechnique, Palaiseau}
\email{vincent.pilaud@lix.polytechnique.fr}
\urladdr{\url{http://www.lix.polytechnique.fr/~pilaud/}}

\date{\today}

\subjclass[2010]{Primary 52B11; Secondary 18M70} 

\keywords{Polytopes...}

\thanks{The first author was supported by the European Union's Horizon 2020 research and innovation program under the Marie Sklodowska-Curie grant agreement No 754362 \inlinegraphics{EU.png}, by the Natural Sciences and Engineering Research Council of Canada (NSERC) and by the ANR-20-CE40-0016 Higher Algebra, Geometry and Topology. The second author...supported by the French ANR grants CAPPS~17\,CE40\,0018, and CHARMS~19\,CE40\,0017.}

%%%%%%%%%%%%%%%%%%%%%%%%%%%%%%%%%%%%%%

\begin{document}

\begin{abstract}
TBC
\end{abstract}

\maketitle

%%%%%%%%%%%%%%%%%%%%%%%%%%%%%%%%%%%%%%%

\section*{Introduction}

Common generalization of \cite{PilaudSignedTree13,LangePilaud13,PonsPilaud18,LA21}.
It is the maximal generalization, according to \cite{Pilaud14}.

\subsection*{Conventions} We write $[n] \eqdef \{1,\ldots,n\}$.

%%%%%%%%%%%%%%%%%%%%%%%%%%%%%%%%%%%%%%%

\section{Combinatorics: the spine complex}

Exemples a suivre tout au long du texte: 1) chemin decore 2) que des down 3) un exemple "generique" de notre cru.

%%%%%%%%%%%%%%%%%%%%%%%%%%%%%%%%%%%%%%%

\subsection{Block graphs}

\begin{definition}
  A \emph{maple tree} is a bicolored tree whose leaves are all of the same color. 
\end{definition}

In the sequel, maple trees will be represented with red and blue colors, and the first one will be used for the leaves.

\begin{definition}
  The \emph{tapping} of a maple tree consists in deleting its red vertices and adjacent edges, and adding edges between all the blue vertices that were adjacent to the same red vertex. 
\end{definition}

\begin{definition}
  \label{def:blockgraph}
  A \emph{block graph} is a connected graph $G$ satisfying any of the following equivalent conditions:
  \begin{enumerate}
    \item Every biconnected component of $G$ is a clique,
    \item Any cycle in $G$ induces a clique,
    \item The intersection of two paths in $G$ induces a path,
    \item The set of connected subgraphs of $G$ is stable under intersection,
    \item $G$ is obtained by tapping a maple tree.
  \end{enumerate}
\end{definition}

\begin{definition}
  A \emph{decoration} of a set of vertices $V$ in a graph $G$ is a bijection $d : V \to X$ for some set $X$ with $|X|=|V|$. 
\end{definition}

\begin{definition}
  A natural number $n$ is \emph{floored} if it is written $n, \up, \down$ or $\updown$.
\end{definition}

\begin{definition}
A \emph{decorated block graph} is a block graph $G$, together with a decoration of its vertices by a choice of floored natural numbers from $1$ to $|V(G)|$.
\end{definition}

From now on we will assume all block graphs to be decorated and we will refer to their set of decorations by $V(G)$.

\begin{definition}
  An \emph{alphanumeric tree} is a maple tree $T$ whose blue vertices $b(T)$ are decorated by distinct letters and whose red vertices $r(T)$ are decorated by floored natural numbers from $1$ to $|r(T)|$.
\end{definition}

The tapping of an alphanumeric tree is a decorated block graph. 

%%%%%%%%%%%%%%%%%%%%%%%%%%%%%%%%%%%%%%%

\subsection{Spine contraction poset}

For a subset of vertices $U$ of a decorated block graph $G$, we use the following notations:
\begin{itemize}
  \item $\up[U] \eqdef \{u \in U \ | \ u=\up \text{ or  } u=\updown \text{ for some } n\}$, 
  \item  $\down[U] \eqdef \{u \in U \ | \ u=\down \text{ or  } u=\updown \text{ for some } n \}$, and 
  \item $\updown[U] \eqdef \up[U] \cap \down[U]$.
\end{itemize}

Let $S$ be a directed tree and let $r$ be a directed edge in $S$. Deleting $r$ separates $S$ into two connected subtrees. We denote by $\so[r]$ (resp. $\ta[r]$) the set of vertices of the connected subtree of $S \setminus \{r\}$ containing the source of $r$ (resp. the target of $r$).

\begin{definition}[Spine] A \emph{spine} on a block graph $G$ is a directed tree $S$ such that
\begin{enumerate}
  \item its set of vertices is decorated by a partition of $V(G)$, and 
  \item at each vertex $U$, the source sets $\so[i_\alpha]$ of the incoming edges $i_{\alpha}$ are contained in distinct connected components of $G \setminus \down[U]$, and the target sets $\ta[o_\beta]$ of the outgoing edges $o_\beta$ are contained in distinct connected components of $G \setminus \up[U]$.
\end{enumerate}
\end{definition}
We denote by $\spines$ the set of spines of a block graph $G$.

\begin{definition}[Spine contraction poset] 
  For two spines $S$ and $S'$ of a block graph $G$, we set $S \leq_G S'$ if and only if any vertex of $S'$ is contained in a vertex of $S$. 
\end{definition}

We now charaterize the covering relations $S'\lessdot_G S$ in this poset. 

\begin{proposition}[Edge contraction] \label{prop:edgecontraction} Let $G$ be a block graph and let $S$ be a spine. A directed tree obtained by contraction of an edge of $S$ is again a spine. Moreover, we have
\[\{ S'\in\mathcal{S}(G) \ | \ S'\lessdot_G S \}=\{\text{spine obtained by contraction of an edge of } S\}\]
\end{proposition}

\begin{figure}[h!]
\centering

\begin{tikzpicture}[scale=1.6]
    
\node (N0) [circle,draw=none,minimum size=4mm,inner sep=0.1mm] at (-0.5,-0.5) {\small $U$};
\node (N1) [circle,draw=none,minimum size=4mm,inner sep=0.1mm] at (0.5,0.5) {\small $V$};

\node (o1) [circle,draw=none,minimum size=4mm,inner sep=0.1mm] at (-1.2,0.6) {\small $o_{1}$};
\node (od) [circle,draw=none,minimum size=4mm,inner sep=0.1mm] at (-0.8,0.6) {\small $\ldots$};
\node (oj) [circle,draw=none,minimum size=4mm,inner sep=0.1mm] at (-0.3,0.6) {\small $o_{j}$};

\node (oj1) [circle,draw=none,minimum size=4mm,inner sep=0.1mm] at (0,1.62) {\small $o_{j+1}$};
\node (ojd) [circle,draw=none,minimum size=4mm,inner sep=0.1mm] at (0.5,1.6) {\small $\ldots$};
\node (ol) [circle,draw=none,minimum size=4mm,inner sep=0.1mm] at (1,1.6) {\small $o_{l}$};

\node (i1) [circle,draw=none,minimum size=4mm,inner sep=0.1mm] at (-1,-1.6) {\small $i_{1}$};
\node (id) [circle,draw=none,minimum size=4mm,inner sep=0.1mm] at (-0.5,-1.6) {\small $\ldots$};
\node (im) [circle,draw=none,minimum size=4mm,inner sep=0.1mm] at (0,-1.6) {\small $i_{m}$};

\node (im1) [circle,draw=none,minimum size=4mm,inner sep=0.1mm] at (0.3,-0.62) {\small $i_{m+1}$};
\node (imd) [circle,draw=none,minimum size=4mm,inner sep=0.1mm] at (0.8,-0.6) {\small $\ldots$};
\node (ik) [circle,draw=none,minimum size=4mm,inner sep=0.1mm] at (1.2,-0.6) {\small $i_{k}$};

\draw[->] (N0)--(o1); 
\draw[->] (N0)--(od);
\draw[->] (N0)--(oj);
  
\draw[->] (N0)--(N1) node[midway,left] {$r$} ; 
\draw[->] (N1)--(oj1); 
\draw[->] (N1)--(ojd);
\draw[->] (N1)--(ol);

\draw[->] (i1)--(N0);
\draw[->] (id)--(N0);
\draw[->] (im)--(N0); 

\draw[->] (im1)--(N1); 
\draw[->] (imd)--(N1);
\draw[->] (ik)--(N1);


\end{tikzpicture}
\quad \quad \resizebox{0.04\linewidth}{!}{\raisebox{5em}{$\longrightarrow$}}\quad \quad
\begin{tikzpicture}[scale=1.2]
    

  \node (N1) [circle,draw=none,minimum size=4mm,inner sep=0.1mm] at (0,0) {\small $U\cup V$};
  
  
  \node (oj1) [circle,draw=none,minimum size=4mm,inner sep=0.1mm] at (-0.6,1.5) {\small $o_{1}$};
  \node (ojd) [circle,draw=none,minimum size=4mm,inner sep=0.1mm] at (0,1.5) {\small $\ldots$};
  \node (ol) [circle,draw=none,minimum size=4mm,inner sep=0.1mm] at (0.6,1.5) {\small $o_{l}$};
  
  
  \node (im1) [circle,draw=none,minimum size=4mm,inner sep=0.1mm] at (-0.6,-1.5) {\small $i_{1}$};
  \node (imd) [circle,draw=none,minimum size=4mm,inner sep=0.1mm] at (0,-1.5) {\small $\ldots$};
  \node (ik) [circle,draw=none,minimum size=4mm,inner sep=0.1mm] at (0.6,-1.5) {\small $i_{k}$};
  
  
  \draw[->] (N1)--(oj1); 
  \draw[->] (N1)--(ojd);
  \draw[->] (N1)--(ol);
  
  \draw[->] (im1)--(N1); 
  \draw[->] (imd)--(N1);
  \draw[->] (ik)--(N1);
  

\end{tikzpicture}
\caption{Contraction of an edge in a spine.}
\label{fig:contraction}
\end{figure} 

\begin{proof} 
  We show that the directed tree $S'$ obtained by contraction of an edge of $S$ is again a spine. The second part of the statement is then straightforward. We give the argument for the incoming edges; the argument for the outgoing edges is symmetric. The situation is represented in \cref{fig:contraction}.
Let $i_\alpha$ and $i_\beta$ be incoming edges of $U$ in $S$. 
\begin{itemize} 
  \item The set $\so[i_\alpha]$ is contained in a connected component $C_1$ of $G\setminus \down[U]$ and also in a connected component $C_2$ of $G\setminus \down[V]$. Thus, it is in $C_1\cap C_2$. By \cref{def:blockgraph}, the subgraph induced by the vertices of $C_1\cap C_2$ is a connected component of $(G \setminus \down[U]) \cap (G \setminus \down[V]) = G \setminus (\down[U]\cup\down[V])$.
  \item Since $\so[i_\alpha]$ and $\so[i_\beta]$ live in two distinct connected components of $G\setminus\down[U]$, they live also in distinct connected components of $G\setminus(\down[U]\cup\down[V])$.
\end{itemize}

Let $i_\alpha$ and $i_\beta$ be incoming edges of $W$ in $S$. 
\begin{itemize} 
  \item Since $\so[i_\alpha]$ is in a connected component of $G \setminus \down[V]$ that does not contain $V$, it is also in a connected component of $G \setminus(\down[U]\cup\down[V])$.
  \item Since $\so[i_\alpha]$ and $\so[i_\beta]$ live in two distinct connected components of $G \setminus \down[V]$, they live also in distinct connected components of $G \setminus (\down[U]\cup\down[V])$.
\end{itemize}

Let $i_\alpha$ be an incoming edge of $U$ in $S$, and let $i_\beta$ be an incoming edge of $V$ in $S$. 
\begin{itemize}
  \item Since $\so[i_\alpha] \subset \so[r]$ and $\so[i_\beta]$ live in two distinct connected components of $G \setminus \down[V]$, they live also in distinct connected components of $G \setminus (\down[U]\cup\down[V])$.
\end{itemize}
\end{proof}

\begin{remark}
  This proposition holds if and only if $G$ is a block graph. 
\end{remark}

\begin{corollary} 
  The poset $\mathcal{S}(G)$ admits a unique minimal element, the spine with only one vertex. 
\end{corollary}

\begin{proposition}[Vertex splitting] 
  \label{prop:vertexsplitting} 
  Let $G$ be a block graph, let $S'$ be a spine on $G$, and consider a vertex $U$ of $S'$. Let $V \sqcup W$ be a partition of $U$ such that
  \begin{enumerate}
    \item $\down[W]$ does not disconnect $V$, and
    \item $\up[V]$ does not disconnect $W$.
  \end{enumerate}
The directed tree $S$ obtained by splitting the vertex $U$ in two vertices $V$ and $W$, adding an edge $r$ from $V$ to $W$ as in \cref{fig:contraction}, and attributing to $V$ and $W$ the edges adjacent to $U$ in the following way
  \begin{enumerate}
    \item[(i)] the incoming edges of $V$ are the connected components of $G\setminus\down[U]$ adjacent to $V$
    \item[(ii)] the outgoing edges of $W$ are the connected components of $G\setminus\up[U]$ adjacent to $W$
  \end{enumerate}
is again a spine. Moreover, we have
  \[\{ S\in\mathcal{S}(G) \ | \ S'\lessdot_G S \}=\{\text{spine obtained by splitting a vertex of } S'\}\]
\end{proposition}

\begin{remark}
  The following proof is "formal", i.e. it doesn't use the fact that $G$ is a block graph. 
\end{remark}

\begin{proof} First note that we can define (i) and (ii) precisely because of the two assumptions (1) and (2). Indeed, saying that $\down[W]$ does not disconnect $V$ is precisely saying that there is a unique connected component of $G\setminus\down[W]$ containing $V$, and similarly for (2) and (ii).
  
We consider only the edges that are adjacent to $W$ in $S$; the argument for the edges adjacent to $V$ in $S$ is symmetric. 
  
Let $o_\alpha$ be an outgoing edge of $W$ in $S$. 
  \begin{itemize}
    \item Since $\ta[o_\alpha]$ is in a connected component of $G \setminus (\up[V]\cup\up[W])$, it is also in a connected component of $G \setminus \up[W]$.
  \end{itemize}
  Let $i_\alpha \neq r$ be an incoming edge of $W$ in $S$. 
  \begin{itemize}
    \item Since $\so[i_\alpha]$ is in a connected component of $G \setminus (\down[V]\cup\down[W])$, it is also in a connected component of $G \setminus \down[W]$.
  \end{itemize}
  Let $r$ be the edge connecting $V$ to $W$ in $S$. We want to show that $\so[r]$ is in a connected component of $G \setminus \down[W]$.
  \begin{itemize}
    
    \item By (2), there is a unique connected component $C$ of $G \setminus {\down[W]}$ that contains $V$. 
    
    \item Let $i_\alpha$ be an incoming edge of $V$ in $S$. Since $\so[i_\alpha]$ is in a connected component of $G \setminus \down[U]$, it is also in a connected component of $G \setminus \down[W]$, which is $C$ by definition (i). 
    
    \item Let $o_\alpha$ be an outgoing edge of $V$ in $S$. Since $\ta[o_\alpha]$ is in a connected component $D$ of $G \setminus \up[U]$, it is also in a connected component $E$ of $G \setminus \up[V]$ and we have $D\subset E$. By definition (ii), $E$ does not contain $W$, which implies in particular that $D \cap \down[W]=\emptyset$, and so $D$ is connected in $G \setminus \down[W]$. Thus, $\ta[o_\alpha]$ is in a connected component of $G \setminus \down[W]$. But it also implies that $D$ is not adjacent to $\up[W]$. So $D$ is adjacent to $\up[V]$ and $\ta[o_\alpha]$ is in the connected component $C$ of $G \setminus \down[W]$. 
  \end{itemize}
  Let $o_\alpha$ and $o_\beta$ be two outgoing edges of $W$ in $S$. 
  \begin{itemize}
    \item Let $C$ and $D$ be the two \emph{distinct} connected components of $G \setminus \up[U]$ containing $\ta[o_\alpha]$ and $\ta[o_\beta]$, respectively. By definition (ii), $C$ and $D$ are both in the connected component of $G \setminus \up[V]$ containing $W$. So, they must be in distinct connected components of $G \setminus \up[W]$. 
  \end{itemize}
  Let $i_\alpha \neq r$ and $i_\beta \neq r$ be two incoming edges of $W$ in S.
  \begin{itemize}
    \item Let $C$ and $D$ be the two \emph{distinct} components of $G\setminus(\down[V]\cup\down[W])$ containing $\so[i_\alpha]$ and $\so[i_\beta]$, respectively. By definition (i), the two connected components $C'$ and $D'$ of $G \setminus \down[W]$ containing $\so[i_\alpha]$ and $\so[i_\beta]$ do not contain $V$. So $C$ and $D$ are not adjacent to $\down[V]$, which implies $C=C'$ and $D=D'$. Thus, $\so[i_\alpha]$ and $\so[i_\beta]$ are in two \emph{distinct} connected components of $G \setminus \down[W]$.
  \end{itemize}
  Let $i_\alpha$ be an incoming edge of $W$ in $S$, and let $r$ be the edge from $V$ to $W$ in $S$.
  \begin{itemize}
    \item The connected component of $G \setminus \down[W]$ containing $\so[i_\alpha]$ does not contain $V$, while the connected component of $G \setminus \down[W]$ containing $\so[r]$ contains $V$, so the two are distinct.
  \end{itemize}
  This concludes the proof of the first part of the statement. For the second part, let $S$ be a spine, and consider an edge from $V$ to $W$ in $S$. It is not hard to see that $V$ and $W$ verify the two conditions (1) and (2), that the incoming edges of $V$ verify property (i) and that the outgoing edges of $W$ verify property (ii). So, at any vertex $U$ of a spine $S'$, the partitions of $U$ verifying (1) and (2) determine \emph{all} the possible spines that can be obtained from $S'$ by splitting the vertex $U$.
\end{proof}
  
In other words, the operations of edge contraction and vertex splitting are inverse to each other.

 
\begin{definition}[Valid partition] 
  Let $G$ be a graph and let $U$ be a subset of vertices. We say that a partition $V \sqcup W$ of $U$ is \emph{valid} if $V$ does not disconnect $W$ and vice-versa.
\end{definition}
  
\begin{lemma} 
  \label{lemma:validpartition} 
  Let $G$ and $G'$ be two graphs such that $V(G)=V(G')$ and $E(G) \subset E(G')$, and let $U \subset V(G)$. If a partition $V \sqcup W$ of $U$ is a valid for $G$, then it is also valid for $G'$.
\end{lemma}

\begin{proof}
  This is straightforward from the definitions. 
\end{proof}
  
\begin{proposition} 
  Let $G$ be a block graph, and let $U$ be a vertex of a spine such that $|U|\geq 2$. Then, $U$ can be splitted. 
\end{proposition}

\begin{proof} 
  We need to consider 3 cases.
  \begin{itemize}
    
    \item If there is $u \in U\setminus\up[U]$, then we can take $V=\{u\}$ and $W = U \setminus \{u\}$.
    
    \item If there is $u \in U\setminus\down[U]$, then we can take $V=U\setminus\{u\}$ and $W=\{u\}$.
    
    \item If $U=\updown[U]$, we claim that there always exists a valid partition of $U$. \cref{lemma:validpartition} shows that it is enough to consider the case when $G$ is a tree. Let $G$ be an oriented tree with at least 2 vertices, and let $U \subset V(G)$ be a set of vertices such that $|U|\geq 2$. Then each edge $r$ of $G$ determines a partition $\so[r] \sqcup \ta[r]$ of $V(G)$. Now pick any edge $r$ in $G$ such that $U \cap \so[r] \neq \emptyset$ and $U \cap \ta[r] \neq \emptyset$. Such an edge always exists since $|U|\geq 2$. Then, $(U \cap \so[r]) sqcup (U \cap \ta[r])$ is a valid partition of $U$.
  
  \end{itemize}
\end{proof}
  
\begin{corollary} 
 The spine poset $\spines$ is graded by the number of vertices. Its maximal elements are the spines for which each vertex has cardinality one. 
\end{corollary}
  
\begin{definition}[Quasi-minimal and maximal spine] 
  Let $G$ be a block graph. A spine $S$ on $G$ is said to be \emph{quasi-minimal} if it has only two vertices, and to be \emph{maximal} if every vertex $U$ satisfies $|U|=1$. We denote the set of maximal spines on $G$ by $\maxspines \subset \spines$. 
\end{definition}
  
%%%%%%%%%%%%%%%%%%%%%%%%%%%%%%%%%%%%%%%

\subsection{Spine flip poset}

\begin{definition}[Flip] Let $G$ be a block graph and let $S \in \maxspines$ be a maximal spine. Consider two vertices $U$ and $V$ of $S$ related by an edge $r$. Now, 
\begin{itemize}
  \item let $i$ be the incoming edge of $U$ associated to the connected component of $G \setminus \down[U]$ containing $V$, and
  \item let $o$ be the outgoing edge of $V$ associated to the connected component of $G \setminus \up[V]$ containing $U$.
\end{itemize}
We define $S'$ to be the spine obtained from $S$ by reversing the orientation of $r$, grafting the edge $i$ to $V$ and the edge $o$ to $U$. We say that $S'$ is obtained from $S$ by a \emph{flip}. 
\end{definition}

\begin{figure}[h!]
  TBC
  \caption{A spine flip.}
\end{figure}

The fact that $S'$ is indeed a spine is immediate from the definitions. Contracting $r$ in either $S$ or $S'$, we obtain the same spine $S''$. 

\begin{lemma} 
  \label{lemma:coveringpair} 
  $S$ and $S'$ form the unique pair of maximal spines covering $S''$ in the spine poset.
\end{lemma}

\begin{proof}
  This follows from \cref{prop:vertexsplitting}.
\end{proof}
  
\begin{corollary} 
   The spine poset $\spines$ is a closed pseudo-manifold. 
\end{corollary}

\begin{proof}
  TBC
\end{proof}
  
\begin{definition}[Flip poset] 
  For $S$ and $S'$ two maximal spines, we say that $S \leq_f S'$ if and only if $S'$ is obtained from $S$ by a flip. 
\end{definition}

\begin{remark}
  The question of whether or not the flip poset $(\mathcal{MS}(G),\leq_f)$ is a lattice is a difficult question. It depends on the decoration of $G$. We know that if $G$ is a tree, it is always the case \cite{PonsPilaud18}, for $G$ a block graph it is generally not true. See \cite{BarnardMcConville18}. \guillaume{Nos calculs?; donner un exemple, un contre-exemple}
\end{remark}
  
\begin{definition}[Flip graph] 
  The \emph{flip graph} is the Hasse diagram of the flip poset. 
\end{definition}


%%%%%%%%%%%%%%%%%%%%%%%%%%%%%%%%%%%%%%%

\section{Geometry: block graph permutreehedra}

%%%%%%%%%%%%%%%%%%%%%%%%%%%%%%%%%%%%%%%

\section{Algebra: operad structure}

%%%%%%%%%%%%%%%%%%%%%%%%%%%%%%%%%%%%%%%

\appendix

\section{More combinatorics}

%%%%%%%%%%%%%%%%%%%%%%%%%%%%%%%%%%%%%%%

\subsection{Nestings, tubings and Forcey-Ronco substitution on graph-associahedra}

%%%%%%%%%%%%%%%%%%%%%%%%%%%%%%%%%%%%%%%

\subsection{Generalized tubings}

%%%%%%%%%%%%%%%%%%%%%%%%%%%%%%%%%%%%%%%

\subsection{Geometric properties of block graph permutreehedra}







\section*{Acknowledgments}

CRM in Barcelona 

\bibliographystyle{alpha}
\bibliography{blockgraph}
\label{sec:biblio}

\end{document}